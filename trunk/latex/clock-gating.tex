\subsection{Clock gating voor het A register}

Aangezien voor het A register gekozen moet worden uit drie  inputs, zijnde A$_{\text{in}}$, A of A$_{\ll 1}$, is voor dit register meer hardware nodig dan voor het B register (waar slechts uit B$_{\text{in}}$ en B gekozen moet worden). In tegenstelling tot de andere registers kan men hier niet anders dan een MUX gebruiken. Ook aan het circuit voor de clock gating moet echter iets veranderd worden.

Allereerst wordt gezocht welk signaal gebruikt moet worden om de MUX te schakelen. Zoals te zien in volgende Karnaugh map, is het antwoord heel eenvoudig:

%\begin{tabular}{r|c|c|}
%	& \multicolumn{2}{|l}{$mode$} \\
%	$start$ & 0 & 1 \\ \hline
%	0 & 1 & 0 \\ \hline
%	1 & 1 & 1 \\ \hline
%\end{tabular}

\karnaughmap{2}{}{{$mode$}{$start$}}{11-0}

Aangezien het bij optellen ($mode = 1$) niet uit maakt welke ingang gekozen wordt tijdens het uitvoeren van de berekening, kan er een "don't care" geplaatst worden op positie 3. Indien we dan A$_{\ll 1}$ aansluiten op de 0-ingang, kan simpelweg $start$ gebruikt worden als schakelsignaal.

Ten tweede zijn er aanpassingen nodig aan het clock gating signaal, indien er vermenigvuldigd wordt, moet er immers elke clock cycle een nieuwe waarde in A opgeslaan worden. Het kloksignaal dient dus doorgelaten te worden indien een berekening begint of indien een vermenigvuldiging aan de gang is. Dit lijdt tot onderstaande Karnaugh map:

\karnaughmap{2}{}{{$mode$}{$start$}}{1011}

M.a.w. het klok signaal aan de ingang van de flip flops wordt:

\[ \text{Clk}_{\text{in}} = start + \overline{mode} \]
