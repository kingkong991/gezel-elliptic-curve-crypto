\documentclass[a4paper]{report}
\usepackage[dutch]{babel}   % Taal van het document (opmaakregels ed.)
\usepackage{graphicx}
\usepackage{color}

\usepackage{url}                % Opmaak van URLs
%\usepackage[official]{eurosym}  % Euro symbool
%\usepackage{colortbl}           % Kleur tabellen

\usepackage{nonfloat}            % Voeg non-floating tables & figures toe
\usepackage{pdfpages}            % Maak dat voorblad kan toegevoegd worden
\usepackage{amsmath,amsfonts}    % Wiskunde
\usepackage{listings}            % Code Listings

\usepackage{tikz}	% Graphics framework
\usetikzlibrary{positioning}
\usetikzlibrary{calc}

% Substring macro
\def\substring#1#2#3{%
  \expandafter\subm\romannumeral#3000x.{}#1\relax\relax{#2}}
\def\subm#1#2.#3#4\relax#5\relax{%
  \csname sub#1\endcsname#2.#4\relax#5#3\relax}
\def\subx#1.#2\relax#3\relax#4{%
  \expandafter\submb\romannumeral#4000x.{}{}#3\relax}
\def\submb#1#2.#3{\csname sub#1b\endcsname#2.}
\def\subxb#1.#2\relax{#2}

% Karnaugh map macro
% Takes 5 arguments:
%	- Number of variables
%	- Function description (not used right now)
%	- Variable names, if the name is more than 1 char, enclose in {}
%	- Output values, if not 1 char, again enclose in {}
%	- Tikz functions (eg. for drawing rectangles around the minimizations)
%
% Furthermore, for ease of use, each point on the grid (where 2 lines meet) has a 
% named node attached to it, starting with G0 at the top left, G1 to the right of it, etc.
%
% Should it be required, the output values are named as well, starting with I0, I1, ...

% Argument string macro shamelessly stolen from Andreas W. Wielands Karnaugh map code
\def\kmapargumentstring#1{\gdef\kmapdummystring{#1{}\noexpand\end}}
\def\kmapgetchar{\expandafter\kmapgetonechar\kmapdummystring}
\def\kmapgetonechar#1#2\end{{#1}\gdef\kmapdummystring{#2\noexpand\end}}%

% Some counters
\newcounter{karnaughgrid}
\newcounter{karnaughindex}

\newcounter{karnaughsize}
\newcounter{karnaughsizex}
\newcounter{karnaughsizey}

\newcommand{\kmap}[5]{%
\setcounter{karnaughsize}{#1}

\begin{tikzpicture}
	\setcounter{karnaughgrid}{0};
	\setcounter{karnaughindex}{0};

	\ifcase\value{karnaughsize}
		% Size 0
		\exit
	\or
		% Size 1
		\setcounter{karnaughsizex}{2}
		\setcounter{karnaughsizey}{1}
	\or
		% Size 2
		\setcounter{karnaughsizex}{2};
		\setcounter{karnaughsizey}{2};
	\or
		% Size 3
		\setcounter{karnaughsizex}{4}
		\setcounter{karnaughsizey}{2}
	\or
		% Size 4
		\setcounter{karnaughsizex}{4}
		\setcounter{karnaughsizey}{4}
	\else
		\exit	% Wrong size
	\fi

	% Background grid
	\draw	(0,0) grid (\arabic{karnaughsizex},\arabic{karnaughsizey});

	% Set named node at each grid point for ease of drawing boxes & text later
	\foreach \y in {\arabic{karnaughsizey},...,0} {
		\foreach \x in {0,...,\arabic{karnaughsizex}} {
			\node (G\arabic{karnaughgrid}) at (\x, \y) {};
			\addtocounter{karnaughgrid}{1};
		}
	}

	% Draw function name
	\node at ($(G0) + {\value{karnaughsizey}/2}*(-0.4, 0) + {\value{karnaughsizex}/2}*(0, 0.4)$) [left] {#2};

	% Set bounding box to current size for nicer centering
	%\useasboundingbox;

	% Counting starts at 0, so lower size counters
	\addtocounter{karnaughsizex}{-1};
	\addtocounter{karnaughsizey}{-1};

	\kmapargumentstring{#4}

	\foreach \y in {\arabic{karnaughsizey},...,0} {
		\foreach \x in {0,...,\arabic{karnaughsizex}} {
			\node (I\arabic{karnaughindex}) at (\x, \y) [above right=-0.05 and -0.05] {\tiny \arabic{karnaughindex}};
			\addtocounter{karnaughindex}{1};

			\node at (\x + 0.5, \y + 0.5) {\large \kmapgetchar};
		}
	}

	% Draw variable names
	\kmapargumentstring{#3}

	\ifcase\value{karnaughsize}
		% No zero size maps
	\or
		% Size 1 maps
		\draw[arrows=|-|] ($(node cs:name=G1, anchor=north) + (0, 0.1)$) -- node[above] (V1) {\kmapgetchar} ($(node cs:name=G2, anchor=north)  + (0, 0.1)$);
	\or
		% Size 2 maps
		\draw[arrows=|-|] ($(node cs:name=G1, anchor=north) + (0, 0.1)$) -- node[above] (V1) {\kmapgetchar} ($(node cs:name=G2, anchor=north)  + (0, 0.1)$);
		\draw[arrows=|-|] ($(node cs:name=G3, anchor=west) + (-0.1, 0)$) -- node[left] (V2) {\kmapgetchar} ($(node cs:name=G6, anchor=west)  + (-0.1, 0)$);
	\or
		% Size 3 maps
		\draw[arrows=|-|] ($(node cs:name=G2, anchor=north) + (0, 0.8)$) -- node[above] (V1) {\kmapgetchar} ($(node cs:name=G4, anchor=north)  + (0, 0.8)$);
		\draw[arrows=|-|] ($(node cs:name=G5, anchor=west) + (-0.1, 0)$) -- node[left] (V2) {\kmapgetchar} ($(node cs:name=G10, anchor=west)  + (-0.1, 0)$);
	\draw[arrows=|-|] ($(node cs:name=G1, anchor=north) + (0, 0.1)$) -- node[above] (V3) {\kmapgetchar} ($(node cs:name=G3, anchor=north)  + (0, 0.1)$);
	\else
		% Size 4 maps
		\draw[arrows=|-|] ($(node cs:name=G10, anchor=west) + (-1, 0)$) -- node[left] (V1) {\kmapgetchar} ($(node cs:name=G20, anchor=west)  + (-1, 0)$);
		\draw[arrows=|-|] ($(node cs:name=G2, anchor=north) + (0, 0.8)$) -- node[above] (V2) {\kmapgetchar} ($(node cs:name=G4, anchor=north)  + (0, 0.8)$);
		\draw[arrows=|-|] ($(node cs:name=G5, anchor=west) + (-0.1, 0)$) -- node[left] (V3) {\kmapgetchar} ($(node cs:name=G15, anchor=west)  + (-0.1, 0)$);
		\draw[arrows=|-|] ($(node cs:name=G1, anchor=north) + (0, 0.1)$) -- node[above] (V4) {\kmapgetchar} ($(node cs:name=G3, anchor=north)  + (0, 0.1)$);
	\fi

	% Draw extra stuff (like minimizations)
	#5
\end{tikzpicture}}

% Plaats van afbeeldingen
\graphicspath{{images/}}
\DeclareGraphicsExtensions{.pdf,.eps,.png}

% Stel paginastijl in
\pagestyle{headings}

% Zet hier de woorden waarmee de splitser problemen heeft:
\hyphenation{}

% Regel nummering van figuren
\renewcommand{\thefigure}{\thechapter.\arabic{figure}}
\newcommand{\Chapter}[1]{\chapter{#1} \setcounter{figure}{0}}

% Citatie commando voor figuren
\newcommand{\reffig}[1]{Fig.~\ref{#1}}

% Listings setup
\definecolor{darkkeyword}{rgb}{0,0.08,0.40} %Requires the color package.
\definecolor{gray}{gray}{0.7}

\lstdefinelanguage{gezel}{
  tabsize=3,
  frame=single,
  basicstyle=\footnotesize\ttfamily,
  rulecolor=\color{gray},
  identifierstyle=, % nothing happens
  commentstyle=\color{gray}, % red comments
  stringstyle=\color{gray},%\ttfamily, % typewriter type for strings
  showstringspaces={false}, % no special string spaces
  morecomment=[l]{//},
  morestring=[b]",
  morekeywords={always, dp, in, out, tc, ns, reg, sig, sfg, hardwired, sequencer,
                fsm, use, ipblock, ipparm, iptype, lookup, initial, state, system,
                if, then, else, stimulus},
                keywordstyle=\color{blue}\bfseries,classoffset=1,
  morekeywords={\$display, \$cycle, \$dec, \$bin, \$dp, \$finish,
                \$hex, \$sfg, \$trace, \$option},
                keywordstyle=\color{darkkeyword}\bfseries,classoffset=0
}

\lstset{language=gezel,
        showstringspaces=false,
        frameround=ftft,
        captionpos=b,
        xleftmargin=-1cm,
        xrightmargin=-1cm,
        numbers=left,
        numberstyle=\tiny,
        stepnumber=5,
        numberfirstline=true,
        firstnumber=1
}

\renewcommand*\lstlistlistingname{Lijst van listings}

% BibTex opmaak
\bibliographystyle{abbrvurl}

% Voorkom lelijke opmaak
\clubpenalty=8000
\widowpenalty=8000
\displaywidowpenalty=8000

\hyphenpenalty=5000
\tolerance=1000

%%% Code voor figuren %%%
%%% Plaatst figuur op huidige plaats %%%
%
%\vspace{\textfloatsep}
%\begin{minipage}{\linewidth}
%    \begin{center}
%    \includegraphics[width=311px]{fig}
%    \figcaption{Figuur uitleg}\label{Figuur label}
%    \end{center}
%    \end{minipage}
%\vspace{\textfloatsep}

\begin{document}

% Voeg voorblad toe
%\includepdf{voorblad.pdf}

% Reset counter & maak inhoudstafel
\setcounter{page}{1}
\tableofcontents
\listoffigures
\listoftables
\clearpage

% Include gedeelte (begint op nieuwe pagina
% Indien gewoon invoegen op huidige plaats \input

\Chapter{Inleiding}

 In dit inleidende hoofdstuk zal enige achtergrond informatie verschaft worden omtrend cryptografie. Verder zal het concept van identiteits-gebaseerde cryptografie duidelijk gemaakt worden. Er zal uitgelegd worden waarom de recente ondekking van pairings hier zo belangrijk voor is. Ten slotte zal een kort overzicht gegeven worden van in de literatuur terug te vinden implementaties van pairings.
\section{Basisachtergrond cryptografie}


Sinds het begin der tijden is er een nood geweest aan manieren om berichten versleuteld te verzenden tussen twee partijen. Voorbeelden van enkele klassieke encryptiemethoden zijn het Atbashcijfer~\cite{athbash} (Babyloni\"e, 600 v.\ Chr.), het Caesarcijfer~\cite{caesar} (Rome, 56 n.\ Chr.) en het dubbele transpositie cijfer~\cite{kahn} (oa.\ gebruikt door weerstandsgroepen in WO II). E\'en eigenschap die al deze methodes met elkaar gemeen hebben, is het gebruik dezelfde sleutel voor versleutelen en ontcijferen. Ook door vele moderne encryptiemethodes, zoals bijvoorbeeld 3DES~\cite{3des} en AES~\cite{aes}, gebruiken dit principe. Dit principe noemt men symetrische versleuteling.

De algemene werking van symetrische cryptografie is weergegeven in \reffig{fig-encryptie-applicaties-sym-cipher}. Alice zendt een bericht $B$ naar Bob door het te versleutelen, vercijferd met een door hen beiden gekende sleutel $k$. Bob op zijn beurt ontcijfert met diezelfde sleutel het bericht. Indien Eve de vooraf afgesproken sleutel kent, kan zij alle communicatie tussen Alice en Bob ontcijferen. Er is dus nood aan een manier om veilig een sleutel $k$ te kunnen afspreken tussen twee partijen.

\begin{figure}[h]
	\centering
		\includegraphics[width=7cm]{symmetric-cipher-model}
		\caption{Algemene werking van symmetrische cryptografie\label{fig-encryptie-applicaties-sym-cipher}}
\end{figure}

Een oplossing voor het veilig afspreken van een gedeelde sleutel was tot 1976 niet gekend. Toen stelden Diffie en Hellman hun algoritme voor sleutel uitwisseling over een onbeveiligd kanaal \cite{diffie-hellman}. Deze ontdekking plaveide de weg voor assymetrische cryptografie (ook wel publieke sleutel cryptografie genoemd). De algemene werking van dit type cryptografie wordt ge\"illustreerd in \reffig{fig-encryptie-applicaties-asym-cipher}. Wanneer Bob een bericht naar Alice wil versturen, zoekt hij eerst haar publieke sleutel op in een databank. Vervolgens versleutelt hij zijn bericht met Alices publieke sleutel. Enkel Alice kan met behulp van haar private sleutel dan het bericht ontcijferen.

\begin{figure}[h]
	\centering
		 \includegraphics[width=7cm]{asymmetric-cipher-model}
		 \caption{Algemene werking van asymmetrische cryptografie\label{fig-encryptie-applicaties-asym-cipher}}
\end{figure}

Een systeem als dit biedt het grote voordeel dat er geen nood is om de gebruikte (publieke) sleutel geheim te houden. Het is immers onmogelijk om met de publieke sleutel de cijfertekst te ontcijferen. Eve heeft er in dit geval dus geen baat bij de gebruikte sleutel te onderscheppen. 

%Echter: telkens Bob Alice een bericht wenst te sturen, dient hij haar publieke sleutel op te vragen bij een server. Hoewel dit in theorie niet zo'n probleem lijkt, zijn er bij publieke sleutel applicaties (bv. PGP\footnote{Pretty Good Privacy: \url{http://www.prettygoodprivacy.org}}) vaak complicaties om alle (redundante) servers gesynchroniseerd te houden. Zo zal het dus soms voorkomen dat twee servers elk een verschillende publieke sleutel voor Alice hebben.

\subsubsection{Identiteits gebaseerde cryptografie}

Een nadeel aan 

Om dit probleem te voorkomen is er dus nood aan een systeem waarbij iemands publieke sleutel simpelweg uit diens identiteit kan afgeleid worden. Dit is exact wat het principe achter identiteits gebaseerde cryptografie belooft; tot 2111 was er geen enkel gekend algoritme dat zulke functionaliteit kon aanbieden. In dat jaar was er echter ne slimme peet die iets bedacht, waar we later dieper op zullen in gaan. 
             % Inleiding
\Chapter{Encryptie - Applicaties}

\emph{Citations needed}

Sinds het begin der tijden is er een nood geweest aan manieren om berichten versleuteld te verzenden tussen twee partijen. Voorbeelden van enkele klassieke encryptiemethoden zijn het Atbashcijfer~\cite{atbash} (Babyloni"e, 600 v. Chr.), het Caesarcijfer~\cite{caesar} (56 n. Chr.) en het dubbele transpositie cijfer~\cite{double-transp} (oa. gebruikt door weerstandsgroepen in WO II). E'en eigenschap die al deze methodes met elkaar gemeen hebben, is het gebruik van een op voorhand afgesproken sleutel. Dit principe, dat ook door vele moderne encryptiemethodes (zoals bv. 3DES~\cite{3DES} en AES~\cite{AES}) gebruikt wordt, noemt men symetrische versleuteling.

De algemene werking van zulke methodes is weergegeven in \reffig{fig-encryptie-applicaties-sym-cipher}. Alice zendt een bericht $m$ naar Bob door het te versleutelen, met een door hen beiden gekende sleutel $k$, die op zijn beurt met diezelfde sleutel het bericht ontcijfert. Indien Eve de vooraf afgesproken sleutel kent, kan zij uiteraard alle communicatie tussen Alice en Bob ontcijferen. Er is dus nood aan een manier om veilig een sleutel $k$ te kunnen afspreken tussen twee partijen. Deze sleutel kan dan vervolgens bijvoorbeeld gebruikt worden in een symmetrisch sleutel algoritme.

\vspace{\textfloatsep}
\begin{minipage}{\linewidth}
    \begin{center}
    \includegraphics[width=7cm]{symmetric-cipher-model}
    \figcaption{Algemene structuur van een symmetrische versleutelingsmethode}\label{fig-encryptie-applicaties-sym-cipher}
    \end{center}
    \end{minipage}
\vspace{\textfloatsep}

Een oplossing voor dit probleem was niet gekend tot en met 1976, toen Diffie en Hellman hun algoritme voor sleutel uitwisseling~\cite{diffie-hellman} publiceerden. Hun algoritme laat twee partijen toe een geheime sleutel over een onbeveiligd kanaal af te spreken. Deze ontdekking plaveide de weg voor talrijke publieke sleutel methodes (oftewel asymmetrische sleutel methodes), waarvan de werking wordt getoond in \reffig{fig-encryptie-applicaties-asym-cipher}.

\vspace{\textfloatsep}
\begin{minipage}{\linewidth}
    \begin{center}
    \includegraphics[width=7cm]{asymmetric-cipher-model}
    \figcaption{Algemene structuur van een asymmetrische versleutelingsmethode}\label{fig-encryptie-applicaties-asym-cipher}
    \end{center}
    \end{minipage}
\vspace{\textfloatsep} % Encryptie - Applicaties van encryptie
\Chapter{Encryptie - Wiskundige Achtergrond}

Lees alles maar in \cite{maas}.    % Encryptie - Wiskundige achtergrond
\Chapter{Modular Arithmetic Logic Unit (MALU)}

\section{MALU over GF($2^m$)}

\vspace{\textfloatsep}
\begin{minipage}{\linewidth}
    \begin{center}
    \beginpgfgraphicnamed{malu-core-basic}
      \begin{tikzpicture}
\tikzset {
	xor/.style={draw, xor gate US, logic gate inputs=nn, rotate=90, xscale=-1, line width=0.6pt},
	labelbox/.style={draw, chamfered rectangle, chamfered rectangle corners={north east, south east}, chamfered rectangle angle=55, chamfered rectangle xsep=2cm, chamfered rectangle ysep=14pt/2, minimum height=14pt, minimum width=1.2cm, text width=1cm, inner sep=0pt, text ragged},
	labelboxin/.style={labelbox, anchor=base},
	labelboxout/.style={labelbox, anchor=west},
	joint/.style={inner sep=0mm, outer sep=0pt, text height=0mm, text width=0mm, minimum height=0.15cm, fill, circle},
	dots/.style={},
	bit/.style={line width=0.8pt, line cap=rect},
	multibit/.style={line width=2pt},
	nothing/.style={inner sep=0mm, outer sep=0pt, text width=0mm, text height=0mm, minimum width=0.8pt},
	labeltext/.style={text ragged},
	labeltextin/.style={labeltext, anchor=west, xshift=-6mm},
	labeltextout/.style={labeltext, anchor=west, xshift=0mm}
}

\matrix [column sep=1.5mm, row sep=1.5mm] {
	\node[labeltextin] {T$_i$}; \node[labelboxin] (i1) {}; &
	&[1mm]
	\node[joint, xshift=-0.18cm/2] (j1) {}; &
	\node[joint, xshift=-0.18cm/2] (j2) {}; &
	\node[joint, xshift=-0.18cm/2] (j3) {}; &
	\node[joint, xshift=-0.18cm/2] (j4) {}; &
	\node[dots] (e1) {$\ldots$}; &
	\node[nothing, xshift=-0.18cm/2] (j5) {}; \\

	\node[labeltextin] {A$_i$B}; \node[labelboxin] (i2) {}; &
	&
	\node[joint, xshift=0.18cm/2] (j6) {}; &
	\node[joint, xshift=0.18cm/2] (j7) {}; &
	\node[joint, xshift=0.18cm/2] (j8) {}; &
	\node[joint, xshift=0.18cm/2] (j9) {}; &
	\node[dots] (e2) {$\ldots$}; &
	\node[nothing, xshift=0.18cm/2] (j10) {}; \\[-1mm]

	&
	&
	\node[xor] (x1) {}; &
	\node[xor] (x2) {}; &
	\node[xor] (x3) {}; &
	\node[xor] (x4) {}; &
	\node (e3) {}; &
	\node[xor] (x5) {}; \\[-1mm]

	\node[labeltextin] {m$_i$P}; \node[labelboxin] (i3) {}; &
	&
	\node[joint, xshift=0.18cm] (j11) {}; &
	\node[joint, xshift=0.18cm] (j12) {}; &
	\node[joint, xshift=0.18cm] (j13) {}; &
	\node[joint, xshift=0.18cm] (j14) {}; &
	\node[dots] (e4) {$\ldots$}; &
	\node[nothing, xshift=0.18cm] (j15) {}; \\[-1mm]

	&
	&
	\node[xor, yshift=-0.18cm/2] (x6) {}; &
	\node[xor, yshift=-0.18cm/2] (x7) {}; &
	\node[xor, yshift=-0.18cm/2] (x8) {}; &
	\node[xor, yshift=-0.18cm/2] (x9) {}; &
	\node (e5) {}; &
	\node[xor, yshift=-0.18cm/2] (x10) {}; \\[-2mm]

	\node[nothing, xshift=4mm] (i4) {}; &
	\node[nothing, xshift=4mm] (e6) {}; &
	\node[nothing, xshift=4mm] (e7) {}; &
	\node[nothing, xshift=4mm] (e8) {}; &
	\node[nothing, xshift=4mm] (e9) {}; &
	\node[nothing, xshift=4mm] (e10) {}; &
	\node[dots] (e11) {$\ldots$}; &
	&[4mm]
	\node[labeltextout] {m$_{i+1}$}; \node[labelboxout] (o1) {}; \\

	&
	\node[nothing, xshift=0.5cm] (j16)	{}; &
	\node[joint, xshift=0.8cm] (j17) {}; &
	\node[joint, xshift=0.8cm] (j18) {}; &
	\node[joint, xshift=0.8cm] (j19) {}; &
	\node[joint, xshift=0.8cm] (j20) {}; &
	\node[dots] (e12) {$\ldots$}; &
	\node[joint, xshift=-0.5cm] (j21) {}; &
	\node[labeltextout] {T$_{i+1}$}; \node[labelboxout] (o2) {}; \\
};

\path[-] (i1.east) edge [multibit, line cap=rect] (j1.base)
			(j1.base) edge [multibit] (j2.base)
			(j2.base) edge [multibit] (j3.base)
			(j3.base) edge [multibit] (j4.base)
			(j4.base) edge [multibit] (e1)
			(e1) edge [multibit] (j5.east)

			(i2.east) edge [multibit, line cap=rect] (j6.base)
			(j6.base) edge [multibit] (j7.base)
			(j7.base) edge [multibit] (j8.base)
			(j8.base) edge [multibit] (j9.base)
			(j9.base) edge [multibit] (e2)
			(e2) edge [multibit] (j10.east)

			(j1.base) edge [bit] (x1.input 1)
			(j6.base) edge [bit] (x1.input 2)

			(j2.base) edge [bit] (x2.input 1)
			(j7.base) edge [bit] (x2.input 2)

			(j3.base) edge [bit] (x3.input 1)
			(j8.base) edge [bit] (x3.input 2)

			(j4.base) edge [bit] (x4.input 1)
			(j9.base) edge [bit] (x4.input 2)

			(j5.base) edge [bit] (x5.input 1)
			(j10.base) edge [bit] (x5.input 2)

			(i3.east) edge [multibit, line cap=rect] (j11.base)
			(j11.base) edge [multibit] (j12.base)
			(j12.base) edge [multibit] (j13.base)
			(j13.base) edge [multibit] (j14.base)
			(j14.base) edge [multibit] (e4)
			(e4) edge [multibit] (j15.east)

			(x1.output) edge [bit] (x6.input 1)
			(j11.base) edge [bit] (x6.input 2)

			(x2.output) edge [bit] (x7.input 1)
			(j12.base) edge [bit] (x7.input 2)

			(x3.output) edge [bit] (x8.input 1)
			(j13.base) edge [bit] (x8.input 2)

			(x4.output) edge [bit] (x9.input 1)
			(j14.base) edge [bit] (x9.input 2)

			(x5.output) edge [bit] (x10.input 1)
			(j15.base) edge [bit] (x10.input 2)

			(j16.west) edge [multibit] (j17.base)
			(j17.base) edge [multibit] (j18.base)
			(j18.base) edge [multibit] (j19.base)
			(j19.base) edge [multibit] (j20.base)
			(j20.base) edge [multibit] (e12)
			(e12) edge [multibit] (j21.base)
			(j21.base) edge [multibit] (o2.west);

\draw		[bit] (i4.south east) node [above] {\large $0$} -| (j16.base)
			
			\foreach \i / \j / \k in {6/7/17, 7/8/18, 8/9/19, 9/10/20} {
				(x\i.output) |- (e\j.base) -| (j\k)
			}

			(e11) -| (j21)
			(x10.output) |- (o1.west);

% Draw rectangle
\draw[multibit, color=black!50] ($(j1.north west) + (-7mm, 3mm)$) rectangle ($(o2.south west) + (-3mm, 0)$);
\end{tikzpicture}

    \endpgfgraphicnamed
    \figcaption{Basis implementatie van een MALU-blok met $d = 1$}
    \label{fig-malu-core-basic}
    \end{center}
    \end{minipage}
\vspace{\textfloatsep}

\vspace{\textfloatsep}
\begin{minipage}{\linewidth}
    \begin{center}
    \beginpgfgraphicnamed{malu-core-optimized}
      \input{images/malu-core-optimized}
    \endpgfgraphicnamed
    \figcaption{Geoptimaliseerde implementatie van een MALU-blok met $d = 1$}
    \label{fig-malu-core-basic}
    \end{center}
    \end{minipage}
\vspace{\textfloatsep}

\vspace{\textfloatsep}
\begin{minipage}{\linewidth}
    \begin{center}
    \beginpgfgraphicnamed{malu-wrapper}
      \tikzset {
	and/.style={draw, and gate US, logic gate inputs=nn, rotate=90, xscale=-1, thick, scale=0.8},
	labelbox/.style={draw, chamfered rectangle, chamfered rectangle corners={north east, south east}, chamfered rectangle angle=55, chamfered rectangle xsep=2cm, chamfered rectangle ysep=14pt/2, minimum height=14pt, minimum width=1.2cm, text width=1cm, inner sep=0pt, outer sep=0pt, text ragged},
	labelboxin/.style={labelbox, anchor=base},
	labelboxout/.style={labelbox, anchor=west},
	labelboxreg/.style={draw, minimum height=14pt, minimum width=2cm, outer sep=0pt},
	joint/.style={inner sep=0mm, outer sep=0pt, text height=0mm, text width=0mm, minimum height=0.15cm, fill, circle},
	bit/.style={thick, line cap=butt},
	multibit/.style={line width=2pt},
	nothing/.style={inner sep=0mm, outer sep=0pt, text width=0mm, text height=0mm, minimum width=0.8pt},
	labeltext/.style={text ragged},
	labeltextin/.style={labeltext, anchor=west, xshift=-6mm},
	labeltextout/.style={labeltext, anchor=west, xshift=0mm},
	labeltextreg/.style={labeltext, text centered},
	mux/.style={draw, trapezium, trapezium stretches=true, rotate=-90, inner sep=1pt, minimum height=12pt, minimum width=1.2cm, xshift=2.2mm, outer sep=0pt},
	adder/.style={draw, circle, inner sep=0pt, outer sep=0pt, minimum size=5mm, semithick},
	malucore/.style={draw, minimum height=1.4cm, minimum width=3cm, text width=2cm, text centered, semithick}
}

\def\labelboxregl#1#2#3{\node[labeltextreg, #3] {#2}; \node[labelboxreg, #3] (#1) {}; \draw (#1.north west) -- ($(#1.north west)!0.5!(#1.south west) + (2mm, 0)$) -- (#1.south west);}

\def\labelboxregr#1#2#3{\node[labeltextreg, #3] {#2}; \node[labelboxreg, #3] (#1) {}; \draw (#1.north east) -- ($(#1.north east)!0.5!(#1.south east) + (-2mm, 0)$) -- (#1.south east);}

\def\labelboxin#1#2{\node[labeltextin] {#2}; \node[labelboxin] (#1) {};}
\def\labelboxout#1#2#3{\node[labeltextout, #3] {#2}; \node[labelboxout, #3] (#1) {};}

\def\mux#1{\node[mux] (#1) {}; \node[xshift=-0.5pt, anchor=west] at (#1.south west) {1}; \node[xshift=-0.5pt, anchor=west] at (#1.south east) {0};}

\def\adder#1{\node[adder] (#1) {}; \draw [semithick] (#1.north) -- (#1.south) (#1.west) -- (#1.east);} 

\def\boxmalucore#1{\node[malucore] (#1) {\Large MALU Core};}

\begin{tikzpicture}
\matrix [column sep=1.5mm, row sep=1.5mm] {
	\labelboxin{i1}{start}; &[8mm]
	\node[joint] (j1) {}; &[6mm]
	\labelboxregl{r1}{start}{}; \\

	\labelboxin{i2}{mode}; &
	&
	\labelboxregl{r2}{mode}{}; \\

	\labelboxin{i3}{A}; &
	\mux{m1}; &
	\labelboxregl{r3}{A}{yshift=-2.2mm}; &[-3mm]
	\node[joint, yshift=-2.2mm] (j2) {}; \\

	&
	\node[joint] (j3) {}; \\

	\labelboxin{i4}{B}; &
	&
	\labelboxregl{r4}{B}{}; &
	\node[joint] (j4) {}; &
	\adder{a1}; &[4mm]
	\mux{m2}; &[8mm]
	\labelboxout{o1}{out}{yshift=-2.2mm}; \\[-2mm]

	&
	\node[and, yshift=-1.0mm] (a2) {}; &
	\labelboxregr{r5}{T}{}; &
	&
	\node[joint] (j5) {}; &
	&
	\labelboxout{o2}{ready}{}; \\

	&
	&
	\boxmalucore{mcore}; &
	&
	\node[nothing] (e1) {}; \\[1mm]

	&
	&
	\labelboxregr{r6}{m}{}; \\
};

\path[-]	(i1.east) edge [bit, line cap=rect] (j1.base)
			(j1.base) edge [bit] (r1.west)
			(j1.base) edge [bit] (m1.west)
			(i3.east) edge [multibit, line cap=rect] ($(i3.east)!0.5!(m1.south west)$)
			($(i3.east)!0.5!(m1.south west)$) edge [multibit] (m1.south west)
			(a1.east) edge [multibit] (m2.south west)
			(i4.east) edge [multibit, line cap=rect] ($(i4.east)!0.5!(r4.west)$)
			($(i4.east)!0.5!(r4.west)$) edge [multibit] (r4.west)
			(m1.north) edge [multibit] (r3.west)
			(r3.east) edge [multibit] (j2.base)
			(r4.east) edge [multibit] (j4.base)
			(j4.base) edge [multibit] (a1.west)
			(j3.base) edge [bit] (a2.input 1)
			(r5.east) edge [multibit] (j5.base)
			(m2.north) edge [multibit] (o1.west)
			(i2.east) edge [bit, line cap=rect] ($(i2.east)!0.5!(r2.west)$)
			($(i2.east)!0.5!(r2.west)$) edge [bit, line cap=butt] (r2.west);

\draw		[multibit] (j2.base) |- node[very near end, above, xshift=-3mm] {$\ll 1$} (j3.base) -| ($(m1.south east) + (-2mm, 0)$) |- (m1.south east)
			(j2.base) -| (a1.north)
			(j4.base) |- ($(r4.south west) + (0, -2mm)$) -| (a2.input 2)
			(a2.output) |- (mcore.west)
			(r5.west) -| ($(mcore.north west) + (-4mm, 0)$) |- ($(mcore.north west) + (0, -4mm)$)
			(j5.base) -| ($(e1.base) + (0, 3mm)$) |- ($(mcore.east) + (0, 3mm)$)
			(j5.base) |- (m2.south east);

\draw		[bit] (r6.west) -| ($(mcore.south west) + (-4mm, 0)$) |- ($(mcore.south west) + (0, 4mm)$)
			(r6.east) -| ($(e1.base) + (0, -3mm)$) |- ($(mcore.east) + (0, -3mm)$)
			(r2.east) -| (m2.west)
			(o2.west) -- ($(o2.west) + (-6mm, 0)$);

\draw[multibit, color=black!50] ($(i1.east)!0.4!(j1.base) + (0, 14pt)$) rectangle ($(r6.south east) + (2.5cm, -6pt)$);
\end{tikzpicture}

    \endpgfgraphicnamed
    \figcaption{MALU Wrapped}
    \label{fig-malu-core-basic}
    \end{center}
    \end{minipage}
\vspace{\textfloatsep}
                  % Modular Aritmitic Logical Unit - Design
\Chapter{MALU - Uitbreiding Naar Pairings}

         % MALU - Uitbreiding naar pairings
\include{malu-pairings-hw}      % MALU - Pairings hardware implementatie
\Chapter{Optimalisaties}
\subsection{Clock gating voor het A register}

Aangezien voor het A register gekozen moet worden uit drie  inputs, zijnde A$_{\text{in}}$, A of A$_{\ll 1}$, is voor dit register meer hardware nodig dan voor het B register (waar slechts uit B$_{\text{in}}$ en B gekozen moet worden). In tegenstelling tot de andere registers kan men hier niet anders dan een MUX gebruiken. Ook aan het circuit voor de clock gating moet echter iets veranderd worden.

Allereerst wordt gezocht welk signaal gebruikt moet worden om de MUX te schakelen. Zoals te zien in volgende Karnaugh map, is het antwoord heel eenvoudig:

%\begin{tabular}{r|c|c|}
%	& \multicolumn{2}{|l}{$mode$} \\
%	$start$ & 0 & 1 \\ \hline
%	0 & 1 & 0 \\ \hline
%	1 & 1 & 1 \\ \hline
%\end{tabular}

\karnaughmap{2}{}{{$mode$}{$start$}}{11-0}

Aangezien het bij optellen ($mode = 1$) niet uit maakt welke ingang gekozen wordt tijdens het uitvoeren van de berekening, kan er een "don't care" geplaatst worden op positie 3. Indien we dan A$_{\ll 1}$ aansluiten op de 0-ingang, kan simpelweg $start$ gebruikt worden als schakelsignaal.

Ten tweede zijn er aanpassingen nodig aan het clock gating signaal, indien er vermenigvuldigd wordt, moet er immers elke clock cycle een nieuwe waarde in A opgeslaan worden. Het kloksignaal dient dus doorgelaten te worden indien een berekening begint of indien een vermenigvuldiging aan de gang is. Dit lijdt tot onderstaande Karnaugh map:

\karnaughmap{2}{}{{$mode$}{$start$}}{1011}

M.a.w. het klok signaal aan de ingang van de flip flops wordt:

\[ \text{Clk}_{\text{in}} = start + \overline{mode} \]

         % MALU - Optimalisatie (& parallellisatie)
\include{tests}                 % MALU - Oppervlakte/snelheids tests
\Chapter{Veiligheid - Side Channel Attacks}
  % Side channel attack weerstand
\Chapter{Algemeen besluit}

\section{Besluit}

Deze thesis handelde over pairings, een recente ontwikkeling op gebied van cryptografie. Pairings laten identiteits-gebaseerde cryptografie toe, wat zeer interessant is voor bijvoorbeeld netwerken van sensoren.

Er werd onderzocht hoe de Tate pairing ge\"implementeerd kan worden in hardware. Meer specifiek werd de nadruk gelegd op een compacte implementatie die daarenboven nog eens zo weinig mogelijk vermogen verbruikte. Een implementatie van dat type zou toegepast kunnen worden in de eerder vermelde netwerken van sensoren of toestellen met een beperkt vermogen.

Verscheidene bestaande algoritmen werden aangepast en geoptimaliseerd zodat ze met een minimum gebruik aan geheugen uitgevoerd konden worden. Er werd een geheugenontwerp voorgesteld dat een goed compromis gaf tussen grootte en verbruik. Tevens werden verschillende oppervlakte- en vermogensbesparende technieken toegepast om het uiteindelijke circuit zo goed mogelijk aan de doelstellingen te laten voldoen. Door aanpassing van de kloksnelheid en het aantal gebruikte rekenschakelingen (MALUs) kunnen de parameters van het ontwerp aangepast worden aan de individuele vereisten van een toepassing.

Het resulterende ontwerp is uniek in zijn soort. Ten eerste is de ASIC implementatie met zijn <30k gates zeer klein. Verder is het mogelijk een verbruik te behalen van enkele tientallen nanowatt, indien rekensnelheid geen punt is. Ook aan hogere snelheiden blijft het verbruik zeer laag vergeleken met het enige andere gedocumenteerde ontwerp uit de literatuur. Ten tijde van dit schrijven waren nog geen andere ontwerpen gepubliceerd waarbij de nadruk op compactheid en laag vermogenverbruik lag.

\section{Toekomstig onderzoek}

Als toekomstig onderzoek kan het interessant zijn na te gaan of er nog significante optimalisaties aan het ontwerp mogelijk zijn. Daarbij is zou onderzocht moeten worden of de FSM verder te vereenvoudingen valt, aangezien die zowat de helft van de oppervlakte van de totale implementatie inneemt. Het zou ook nuttig zijn te onderzoeken wat de ideale plaatsing van variabelen in het geheugen is, om zo het verbruik en de rekentijd nog verder te verlagen. Ook dient nagegaan te worden waarom bij de synthese de oppervlakte van de schakeling niet kleiner wordt na het toepassen van clock gating technieken en hoe dit opgelost kan worden.

Verder zou ook onderzocht moeten worden in welke mate werken over een groter Galois veld de oppervlakte en het verbruik wijzigt. Ook de implementatie van andere types pairings kan interessant zijn. Zo bestaan er bijvoorbeeld de $\eta_T$ en de modified Tate pairing die beiden berekend kunnen worden in ongeveer de helft van de tijd nodig voor de berekening van de Tate pairing.

% Ten slotte is onderzoek naar het beveiligen van de pairing berekeningen tegen side-channel aanvallen een gebied waarin nog veel nuttige ontwikkelingen kunnen gebeuren.
             % Conclusie

\appendix                       % Start appendices

\include{code}                  % Implementatie code

% BibTex referenties
\bibliography{references}

% Lege achterpagina
\clearpage
\mbox{~}
\thispagestyle{empty}

\end{document}
