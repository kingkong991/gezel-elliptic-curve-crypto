% To check for overfull hboxes, add draft to the document options
\documentclass[a4paper, 11pt]{report}
\usepackage[dutch]{babel}   % Taal van het document (opmaakregels ed.)
\usepackage{fancyhdr} % Fancy headers & footers

\usepackage{graphicx}
\usepackage{subfigure}	% Multiple figures in one float

\usepackage{color}
\usepackage{colortbl}	% Use colors in table
\usepackage{multirow}	% Multiple rows and columns in tables
\usepackage{booktabs}	% Professional looking tables

\usepackage{amsmath,amsfonts}    % Wiskunde

\usepackage{listings}            % Code Listings
%\usepackage[subfigure]{tocloft}		% Lijsten van willekeurige dingen + optie om errors met subfigure te vermijden
\usepackage[ruled,vlined,linesnumbered]{algorithm2e}	% Algorithm typesetting
%\usepackage{url}                % Opmaak van URLs

\usepackage{pdfpages}            % Maak dat voorblad kan toegevoegd worden
\usepackage[pdftex, pdfborder={0 0 0}]{hyperref}	% Links in pdf

\usepackage{ifthen}	% Conditionele stuff
\newboolean{makeall}	% Als dit true is, wordt inhoudsopgave etc toegevoegd
\newboolean{dankwoord}
\newboolean{spacepar}	% Spatie tussen paragrafen in plaats van indentatie
\setboolean{makeall}{true}
\setboolean{dankwoord}{true}
\setboolean{spacepar}{false}

%\usepackage{tikz}	% Graphics framework
\usetikzlibrary{positioning}
\usetikzlibrary{calc}

% Karnaugh map macro
% Takes 5 arguments:
%	- Number of variables
%	- Function description (not used right now)
%	- Variable names, if the name is more than 1 char, enclose in {}
%	- Output values, if not 1 char, again enclose in {}
%	- Tikz functions (eg. for drawing rectangles around the minimizations)
%
% Furthermore, for ease of use, each point on the grid (where 2 lines meet) has a 
% named node attached to it, starting with G0 at the top left, G1 to the right of it, etc.
%
% Should it be required, the output values are named as well, starting with I0, I1, ...

% Argument string macro shamelessly stolen from Andreas W. Wielands Karnaugh map code
\def\kmapargumentstring#1{\gdef\kmapdummystring{#1{}\noexpand\end}}
\def\kmapgetchar{\expandafter\kmapgetonechar\kmapdummystring}
\def\kmapgetonechar#1#2\end{{#1}\gdef\kmapdummystring{#2\noexpand\end}}%

% Some counters
\newcounter{karnaughgrid}
\newcounter{karnaughindex}

\newcounter{karnaughsize}
\newcounter{karnaughsizex}
\newcounter{karnaughsizey}

\newcommand{\kmap}[5]{%
\setcounter{karnaughsize}{#1}

\begin{tikzpicture}
	\setcounter{karnaughgrid}{0};
	\setcounter{karnaughindex}{0};

	\ifcase\value{karnaughsize}
		% Size 0
		\exit
	\or
		% Size 1
		\setcounter{karnaughsizex}{2}
		\setcounter{karnaughsizey}{1}
	\or
		% Size 2
		\setcounter{karnaughsizex}{2};
		\setcounter{karnaughsizey}{2};
	\or
		% Size 3
		\setcounter{karnaughsizex}{4}
		\setcounter{karnaughsizey}{2}
	\or
		% Size 4
		\setcounter{karnaughsizex}{4}
		\setcounter{karnaughsizey}{4}
	\else
		\exit	% Wrong size
	\fi

	% Background grid
	\draw	(0,0) grid (\arabic{karnaughsizex},\arabic{karnaughsizey});

	% Set named node at each grid point for ease of drawing boxes & text later
	\foreach \y in {\arabic{karnaughsizey},...,0} {
		\foreach \x in {0,...,\arabic{karnaughsizex}} {
			\node (G\arabic{karnaughgrid}) at (\x, \y) {};
			\addtocounter{karnaughgrid}{1};
		}
	}

	% Draw function name
	\node at ($(G0) + {\value{karnaughsizey}/2}*(-0.4, 0) + {\value{karnaughsizex}/2}*(0, 0.4)$) [left] {#2};

	% Set bounding box to current size for nicer centering
	%\useasboundingbox;

	% Counting starts at 0, so lower size counters
	\addtocounter{karnaughsizex}{-1};
	\addtocounter{karnaughsizey}{-1};

	\kmapargumentstring{#4}

	\foreach \y in {\arabic{karnaughsizey},...,0} {
		\foreach \x in {0,...,\arabic{karnaughsizex}} {
			\node (I\arabic{karnaughindex}) at (\x, \y) [above right=-0.05 and -0.05] {\tiny \arabic{karnaughindex}};
			\addtocounter{karnaughindex}{1};

			\node at (\x + 0.5, \y + 0.5) {\large \kmapgetchar};
		}
	}

	% Draw variable names
	\kmapargumentstring{#3}

	\ifcase\value{karnaughsize}
		% No zero size maps
	\or
		% Size 1 maps
		\draw[arrows=|-|] ($(node cs:name=G1, anchor=north) + (0, 0.1)$) -- node[above] (V1) {\kmapgetchar} ($(node cs:name=G2, anchor=north)  + (0, 0.1)$);
	\or
		% Size 2 maps
		\draw[arrows=|-|] ($(node cs:name=G1, anchor=north) + (0, 0.1)$) -- node[above] (V1) {\kmapgetchar} ($(node cs:name=G2, anchor=north)  + (0, 0.1)$);
		\draw[arrows=|-|] ($(node cs:name=G3, anchor=west) + (-0.1, 0)$) -- node[left] (V2) {\kmapgetchar} ($(node cs:name=G6, anchor=west)  + (-0.1, 0)$);
	\or
		% Size 3 maps
		\draw[arrows=|-|] ($(node cs:name=G2, anchor=north) + (0, 0.8)$) -- node[above] (V1) {\kmapgetchar} ($(node cs:name=G4, anchor=north)  + (0, 0.8)$);
		\draw[arrows=|-|] ($(node cs:name=G5, anchor=west) + (-0.1, 0)$) -- node[left] (V2) {\kmapgetchar} ($(node cs:name=G10, anchor=west)  + (-0.1, 0)$);
	\draw[arrows=|-|] ($(node cs:name=G1, anchor=north) + (0, 0.1)$) -- node[above] (V3) {\kmapgetchar} ($(node cs:name=G3, anchor=north)  + (0, 0.1)$);
	\else
		% Size 4 maps
		\draw[arrows=|-|] ($(node cs:name=G10, anchor=west) + (-1, 0)$) -- node[left] (V1) {\kmapgetchar} ($(node cs:name=G20, anchor=west)  + (-1, 0)$);
		\draw[arrows=|-|] ($(node cs:name=G2, anchor=north) + (0, 0.8)$) -- node[above] (V2) {\kmapgetchar} ($(node cs:name=G4, anchor=north)  + (0, 0.8)$);
		\draw[arrows=|-|] ($(node cs:name=G5, anchor=west) + (-0.1, 0)$) -- node[left] (V3) {\kmapgetchar} ($(node cs:name=G15, anchor=west)  + (-0.1, 0)$);
		\draw[arrows=|-|] ($(node cs:name=G1, anchor=north) + (0, 0.1)$) -- node[above] (V4) {\kmapgetchar} ($(node cs:name=G3, anchor=north)  + (0, 0.1)$);
	\fi

	% Draw extra stuff (like minimizations)
	#5
\end{tikzpicture}}
	% Karnaugh map macros

%\usepackage{tikz}
%\pgfrealjobname{thesis}

% Lengtes
%\addtolength{\parskip}{10pt}
%\addtolength{\intextsep}{10pt}
%\addtolength{\belowcaptionskip}{-10pt}

% Hyphenation
%\hyphenation{}

% Math stuff
\newcommand{\xor}{\oplus}

% Citatie commando's
\newcommand{\reffig}[1]{Figuur~\ref{#1}}
\newcommand{\reftbl}[1]{Tabel~\ref{#1}}
\newcommand{\refalg}[1]{Algoritme~\ref{#1}}
\newcommand{\refsect}[1]{Paragraaf~\ref{#1}}
\newcommand{\refhfdst}[1]{Hoofdstuk~\ref{#1}}
\newcommand{\refform}[1]{Formule~\ref{#1}}

% Misc commando's
\newcommand{\nr}{n$^{\circ}$~}
%\def\@fnsymbol#1{\ifcase#1\or *\or \dagger\or \ddagger\or \mathchar "278\or \mathchar "27B\or \|\or **\or \dagger\dagger \or \ddagger\ddagger \else\@ctrerr\fi\relax}
\renewcommand{\thefootnote}{\fnsymbol{footnote}}	% Voetnoten met symbolen ipv nummers

% Narrow environment for wide tables & figures
% Usage: \begin{narrow}{-left cm}{-right cm}
\newenvironment{narrow}[2]{%
  \begin{list}{}{%
    \setlength{\topsep}{0pt}%
    \setlength{\leftmargin}{#1}%
    \setlength{\rightmargin}{#2}%
    \setlength{\listparindent}{\parindent}%
    \setlength{\itemindent}{\parindent}%
    \setlength{\parsep}{\parskip}%
  }%
  \item[]
}{\end{list}}

% List of equations
%\newcommand{\listequationsname}{List of vergelijkingen}
%\newlistof{myequations}{equ}{\listequationsname}
%\newcommand{\myequations}[1]{%
%\addcontentsline{equ}{myequations}{\protect\numberline{\theequation}#1}\par}

% BibTex opmaak
%\bibliographystyle{abbrvurl}
%\bibliographystyle{abbrv}
%\bibliographystyle{plain}
%\bibliographystyle{IEEEtranS} % Sorted IEEE style
\bibliographystyle{IEEEtran} % Unsorted IEEE style

% Fix for unsorted bibliographies & citations in captions
\def\@starttoc#1{%
  \begingroup
    \@fileswfalse
    \makeatletter
    \@input{\jobname.#1}%
  \endgroup
  \if@filesw
    \expandafter\newwrite\csname tf@#1\endcsname
    \immediate\openout \csname tf@#1\endcsname \jobname.#1\relax
  \fi
  \@nobreakfalse
}

% Substring macro
\def\substring#1#2#3{%
  \expandafter\subm\romannumeral#3000x.{}#1\relax\relax{#2}}
\def\subm#1#2.#3#4\relax#5\relax{%
  \csname sub#1\endcsname#2.#4\relax#5#3\relax}
\def\subx#1.#2\relax#3\relax#4{%
  \expandafter\submb\romannumeral#4000x.{}{}#3\relax}
\def\submb#1#2.#3{\csname sub#1b\endcsname#2.}
\def\subxb#1.#2\relax{#2}

% Plaats van afbeeldingen
\graphicspath{{images/}}
\DeclareGraphicsExtensions{.pdf,.eps,.png}

% Stel headers & footers in
\fancyhfoffset[EO]{2.5cm}

\fancyhead[LE, LO]{\small \slshape \rightmark}
\fancyhead[RE, RO]{\small \slshape \leftmark}

% Voor afwisselende pagina's:
%\fancyhead[LE,RO]{\small \slshape \rightmark}
%\fancyhead[RE,LO]{\small \slshape \leftmark}

\fancyfoot[CE,CO]{\thepage}

\pagestyle{fancy}

% Fix caption spacing bij tabellen
\setlength{\belowcaptionskip}{6pt}

% Zet hier de woorden waarmee de splitser problemen heeft:
\hyphenation{}

% Regel nummering van figuren
\renewcommand{\thefigure}{\thechapter.\arabic{figure}}
\newcommand{\Chapter}[1]{\chapter{#1} \setcounter{figure}{0}}

% Cell coloring
\newcommand{\shadecell}{\cellcolor{black!65}}

% Algorithm2e setup
\renewcommand{\listalgorithmcfname}{Lijst van algoritmes}
\renewcommand{\algorithmcfname}{Algoritme}
\renewcommand{\thealgocf}{\thechapter.\arabic{algocf}}
\SetCommentSty{textsf}
\SetKwComment{Comment}{}{}
\newcommand{\comm}[1]{\Comment*[f]{#1}}

% Listings setup
\definecolor{darkkeyword}{rgb}{0,0.08,0.40} %Requires the color package.
\definecolor{gray}{gray}{0.7}

\lstdefinelanguage{gezel}{
  tabsize=3,
  frame=single,
  basicstyle=\footnotesize\ttfamily,
  rulecolor=\color{gray},
  identifierstyle=, % nothing happens
  commentstyle=\color{gray}, % red comments
  stringstyle=\color{gray},%\ttfamily, % typewriter type for strings
  showstringspaces={false}, % no special string spaces
  morecomment=[l]{//},
  morestring=[b]",
  morekeywords={always, dp, in, out, tc, ns, reg, sig, sfg, hardwired, sequencer,
                fsm, use, ipblock, ipparm, iptype, lookup, initial, state, system,
                if, then, else, stimulus},
                keywordstyle=\color{blue}\bfseries,classoffset=1,
  morekeywords={\$display, \$cycle, \$dec, \$bin, \$dp, \$finish,
                \$hex, \$sfg, \$trace, \$option},
                keywordstyle=\color{darkkeyword}\bfseries,classoffset=0
}

\lstset{language=gezel,
        showstringspaces=false,
        frameround=ftft,
        captionpos=b,
        xleftmargin=-1cm,
        xrightmargin=-1cm,
        numbers=left,
        numberstyle=\tiny,
        stepnumber=5,
        numberfirstline=true,
        firstnumber=1
}

\renewcommand*\lstlistlistingname{Lijst van listings}

% Voorkom lelijke opmaak
%\clubpenalty=8000
%\widowpenalty=8000
%\displaywidowpenalty=8000

%\hyphenpenalty=5000
%\tolerance=1000

%%% Code voor figuren %%%
%%% Plaatst figuur op huidige plaats %%%
%
%\vspace{\textfloatsep}
%\begin{minipage}{\linewidth}
%    \begin{center}
%    \includegraphics[width=311px]{fig}
%    \figcaption{Figuur uitleg}\label{Figuur label}
%    \end{center}
%    \end{minipage}
%\vspace{\textfloatsep}

\begin{document}
\dontprintsemicolon   % Don't print ; after each line in algorithms
\selectlanguage{dutch}  % Set hyphenation patterns
\hyphenation{}

% Include gedeelte (begint op nieuwe pagina
% Indien gewoon invoegen op huidige plaats \input

\ifthenelse{\boolean{makeall}}%
{
\pagestyle{empty}
\pagenumbering{roman}

% Voeg voorblad toe
%\includepdf{voorblad-avanherr.pdf}
\includepdf{voorblad.pdf}

% Copyright
\section*{}
\thispagestyle{empty}

\vfill

Copyright K.U.\ Leuven

\bigskip \noindent Zonder voorafgaande schriftelijke toestemming van zowel de promotor(en) als de auteur(s) is overnemen, kopi\"eren, gebruiken of realiseren van deze uitgave of gedeelten ervan verboden. 

\medskip \noindent Voor aanvragen tot, of informatie i.v.m.\ het overnemen en/of gebruik en/of realisatie van gedeelten uit deze publicatie, wend U tot de K.U.\ Leuven, Departement Elektrotechniek - ESAT, Kasteelpark Arenberg 10, BE-3001 Heverlee (Belgi\"e). Telefoon \mbox{+32 16 32 11 30} \& Fax.\ \mbox{+32 16 32 19 86}

\medskip \noindent Voorafgaande schriftelijke toestemming van de promotor(en) is eveneens vereist voor het aanwenden van de in dit afstudeerwerk beschreven (originele) methoden, producten, schakelingen en programma's voor industrieel of commercieel nut en voor de inzending van deze publicatie ter deelname aan wetenschappelijke prijzen of wedstrijden.

\bigskip
\bigskip
\bigskip

\noindent Copyright by K.U.\ Leuven

\bigskip \noindent Without written permission of the promotors and the authors it is forbidden to reproduce or adapt in any form or by any means any part of this publication. 

\medskip \noindent Requests for obtaining the right to reproduce or utilize parts of this publication should be addressed to K.U.\ Leuven, Departement Elektrotechniek - ESAT, Kasteelpark Arenberg 10,  BE-3001 Heverlee  (Belgium). Tel.\ \mbox{+32 16 32 11 30} \& Fax.\ \mbox{+32 16 32 19 86}.

\medskip \noindent A written permission of the promotor is also required to use the methods, products, schematics and programs described in this work for industrial or commercial use, and for submitting this publication to scientific contests.

\clearpage


\ifthenelse{\boolean{dankwoord}}{
\chapter*{Voorwoord}

Graag zou ik mijn promotoren Bart Preneel en Ingrid Verbauwhede bedanken voor het mogelijk maken van deze thesis. Indien zij er niet geweest waren, zou het nooit bij me opgekomen zijn een thesis over dit onderwerp te maken.

Verder wil ik graag mijn dagelijkse begeleiders Lejla Batina en Miroslav Knezevic bedanken voor hun steun en vele raad die ze mij gaven. Lejla wil ik specifiek bedanken voor de vele papers die ze mij bezorgde en haar hulp bij verscheidene wiskundige struikelblokken die ik tegenkwam. Miroslav dank ik voor zijn inzicht i.v.m.\ hardware en de vele tijd die hij stak in het helpen synthetiseren van mijn uiteindelijke ontwerp.

Ook dr.\ ir.\ Frederik Vercauteren verdient speciale vermelding. Zijn vermogen om de ingewikkelde wiskunde achter cryptografie zo uit te leggen dat zelfs een klein kind het nog zou begrijpen, heeft mij meer inzicht in de materie gegeven. Ook zijn hulp met bepaalde software en zijn niet aflatende vrolijkheid apprecieerde ik ten zeerste.

Dr.\ ir.\ Nele Mentens verdient als nalezer ook een bedanking in dit voorwoord. Zonder haar kritisch oog was deze tekst vast en zeker niet geworden wat hij nu is.

Mijn vrienden wil ik graag bedanken voor hun steun en de vele fijne momenten die we samen beleefden dit jaar. Zonder hen zou het afgelopen jaar ongetwijfeld een stuk minder memorabel geweest zijn. Ook wil ik specifiek Mathias De Somere bedanken voor zijn hulp bij het oplossen van (zowat steeds compleet irrelevante) wiskundige problemen.

Dirk en Martine wil ik bedanken voor alles wat ze voor mij gedaan hebben. Ten slotte wil ik mijn ouders bedanken. Zij hebben het voor mij mogelijk gemaakt de studies Burgerlijk Ingenieur te ondernemen en zijn me gedurende die hele tijd steeds onvoorwaardelijk blijven steunen. 

\bigskip \bigskip
{\raggedleft	% Lijn rechts uit
Anthony Van Herrewege\\
22 mei 2009\\
}
  % Dankwoord
}{}

\chapter*{Samenvatting}

In deze thesis wordt de berekening van de Tate paring onder de loep genomen. Een paring is een wiskundige constructie waarvan in 2001 ontdekt werd dat ze gebruikt kan worden voor het implementeren van identiteits-gebaseerde cryptografie.

Er wordt een compact hardware ontwerp voorgesteld dat de Tate paring over een supersinguliere kromme in $\mathbb{F}_{2^{163}}$ op een zeer zuinige manier kan berekenen. Het geheugengebruik van de ge\"implementeerde algoritmes wordt geminimaliseerd en een ontwerp voor een effici\"ent geheugenblok passeert de revue. Ook wordt het effect van verscheidene vermogensoptimalisaties onderzocht.

De uiteindelijke schakeling neemt minimum ongeveer 30k gates in beslag, dit is meer dan drie keer kleiner dan het kleinste ontwerp uit de literatuur. Een vermogenverbruik zo laag als 206 nanowatt kan bereikt worden, hoewel in dat geval de rekentijd 51,5 seconden bedraagt. Indien gewenst kan de berekening versneld worden door uitbreiding van de rekenkern. De energie-effici\"entie is, afhankelijk van gekozen parameters, tussen twee en twintig maal beter dan die van de enige andere implementatie uit de literatuur waarvoor dit berekend kon worden.

Ten tijde van dit schrijven zijn nog geen andere compacte, energie-effici\"ente ontwerpen gepubliceerd. Het voorgestelde ontwerp is in dat opzicht dus uniek.
 % Samenvatting

\tableofcontents
\clearpage

\listoffigures
\clearpage

\listoftables
\clearpage
  
\listofalgorithms
\clearpage

\chapter*{Lijst van afkortingen}

\begin{tabular}{l@{$\qquad$}l}
ASIC	& Application-Specific Integrated Circuit\\
ECC	& Elliptic Curve Cryptography\\
FPGA	& Field-Programmable Gate Array\\
MALU	& Modulaire Arithmetische Logische Unit\\
MUX	& Multiplexer\\
RAM	& Random-Access Memory\\
VHDL	& Very high speed integrated circuits Hardware Description Language\\
\end{tabular}
 % Lijst van afkortingen
\chapter*{Lijst van symbolen}

\section*{Algemeen}

\begin{tabular}{l@{$\qquad$}l}
$\mathbb{F}$	& Galois veld\\
$\mathbb{F}_q$	& Galois veld van $q$ elementen\\
%$m$				& Graad van het Galois veld $\mathbb{F}_{2^m}$\\
$E$				& Elliptische kromme\\
$\#E$				& Aantaal punten op de elliptische kromme $E$\\
$x_A$				& x-co\"ordinaat van het punt $A$\\
$y_A$				& y-co\"ordinaat van het punt $A$\\
$e(P, Q)$		& Tate pairing van $P$ en $Q$\\
$\textsf{degree}(a)$	& Graad van de veelterm $a$\\
$\textsf{Hamm}(a)$	& Hamming gewicht van binaire representatie van $a$\\
$\textsf{min}(a, b)$	& Minimum  uit de lijst van opgegeven argumenten ($a$ en $b$)\\
\end{tabular}

\section*{Operators}

\begin{tabular}{l@{$\qquad$}l}
$\xor$	&	exclusieve OR\\
$\#$		&	concatenatie\\
\end{tabular}
    % Lijst van symbolen
}{}

\ifthenelse{\boolean{spacepar}}{
	% Paragraph spacing
	\setlength{\parindent}{0pt}
	\setlength{\parskip}{2ex plus 0.5ex minus 0.2ex}
}{}

\pagestyle{fancy}
\pagenumbering{arabic}

\Chapter{Inleiding}

 In dit inleidende hoofdstuk zal enige achtergrond informatie verschaft worden omtrend cryptografie. Verder zal het concept van identiteits-gebaseerde cryptografie duidelijk gemaakt worden. Er zal uitgelegd worden waarom de recente ondekking van pairings hier zo belangrijk voor is. Ten slotte zal een kort overzicht gegeven worden van in de literatuur terug te vinden implementaties van pairings.
\section{Basisachtergrond cryptografie}


Sinds het begin der tijden is er een nood geweest aan manieren om berichten versleuteld te verzenden tussen twee partijen. Voorbeelden van enkele klassieke encryptiemethoden zijn het Atbashcijfer~\cite{athbash} (Babyloni\"e, 600 v.\ Chr.), het Caesarcijfer~\cite{caesar} (Rome, 56 n.\ Chr.) en het dubbele transpositie cijfer~\cite{kahn} (oa.\ gebruikt door weerstandsgroepen in WO II). E\'en eigenschap die al deze methodes met elkaar gemeen hebben, is het gebruik dezelfde sleutel voor versleutelen en ontcijferen. Ook door vele moderne encryptiemethodes, zoals bijvoorbeeld 3DES~\cite{3des} en AES~\cite{aes}, gebruiken dit principe. Dit principe noemt men symetrische versleuteling.

De algemene werking van symetrische cryptografie is weergegeven in \reffig{fig-encryptie-applicaties-sym-cipher}. Alice zendt een bericht $B$ naar Bob door het te versleutelen, vercijferd met een door hen beiden gekende sleutel $k$. Bob op zijn beurt ontcijfert met diezelfde sleutel het bericht. Indien Eve de vooraf afgesproken sleutel kent, kan zij alle communicatie tussen Alice en Bob ontcijferen. Er is dus nood aan een manier om veilig een sleutel $k$ te kunnen afspreken tussen twee partijen.

\begin{figure}[h]
	\centering
		\includegraphics[width=7cm]{symmetric-cipher-model}
		\caption{Algemene werking van symmetrische cryptografie\label{fig-encryptie-applicaties-sym-cipher}}
\end{figure}

Een oplossing voor het veilig afspreken van een gedeelde sleutel was tot 1976 niet gekend. Toen stelden Diffie en Hellman hun algoritme voor sleutel uitwisseling over een onbeveiligd kanaal \cite{diffie-hellman}. Deze ontdekking plaveide de weg voor assymetrische cryptografie (ook wel publieke sleutel cryptografie genoemd). De algemene werking van dit type cryptografie wordt ge\"illustreerd in \reffig{fig-encryptie-applicaties-asym-cipher}. Wanneer Bob een bericht naar Alice wil versturen, zoekt hij eerst haar publieke sleutel op in een databank. Vervolgens versleutelt hij zijn bericht met Alices publieke sleutel. Enkel Alice kan met behulp van haar private sleutel dan het bericht ontcijferen.

\begin{figure}[h]
	\centering
		 \includegraphics[width=7cm]{asymmetric-cipher-model}
		 \caption{Algemene werking van asymmetrische cryptografie\label{fig-encryptie-applicaties-asym-cipher}}
\end{figure}

Een systeem als dit biedt het grote voordeel dat er geen nood is om de gebruikte (publieke) sleutel geheim te houden. Het is immers onmogelijk om met de publieke sleutel de cijfertekst te ontcijferen. Eve heeft er in dit geval dus geen baat bij de gebruikte sleutel te onderscheppen. 

%Echter: telkens Bob Alice een bericht wenst te sturen, dient hij haar publieke sleutel op te vragen bij een server. Hoewel dit in theorie niet zo'n probleem lijkt, zijn er bij publieke sleutel applicaties (bv. PGP\footnote{Pretty Good Privacy: \url{http://www.prettygoodprivacy.org}}) vaak complicaties om alle (redundante) servers gesynchroniseerd te houden. Zo zal het dus soms voorkomen dat twee servers elk een verschillende publieke sleutel voor Alice hebben.

\subsubsection{Identiteits gebaseerde cryptografie}

Een nadeel aan 

Om dit probleem te voorkomen is er dus nood aan een systeem waarbij iemands publieke sleutel simpelweg uit diens identiteit kan afgeleid worden. Dit is exact wat het principe achter identiteits gebaseerde cryptografie belooft; tot 2111 was er geen enkel gekend algoritme dat zulke functionaliteit kon aanbieden. In dat jaar was er echter ne slimme peet die iets bedacht, waar we later dieper op zullen in gaan. 
       % Inleiding & motivatie

\Chapter{Pairings}

In dit hoofdstuk zal de werking van pairings wiskundig uit de doeken gedaan worden. Meer specifiek zal de Tate pairing bestudeerd worden. Er zal duidelijk gemaakt worden hoe de pairing berekend kan worden. Vervolgens worden enkele schema's voorgesteld die gebruikt kunnen worden voor versleuteling van gegevens en de aanmaak van digitale handtekeningen. In het volgende hoofdstuk wordt dan een schakeling ontworpen waarmee de Tate pairing berekend kan worden.

Enkel de hoogstnodige theorie zal hier behandeld worden. Voor een meer diepgaande uiteenzetting wordt opnieuw verwezen naar \cite{maas}. Het is ook uit dit werk dat de informatie uit de volgende paragrafen afkomstig is, tenzij anders vermeld.

\section{Inleidende wiskunde}

Alvorens de wiskundige theorie van pairings uit de doeken gedaan kan worden, dient die van elliptische krommen duidelijk gemaakt te worden. Het is met behulp van deze constructies dat pairings berekend kunnen worden. Elliptische krommen worden doorgaans echter gedefinieerd over een eindig veld. Vandaar dat de noodzakelijk theorie van beiden hier even heel kort herhaald wordt.

\subsection{Eindige velden}

Een eindig veld $\mathbb{F}_q$ wordt gedefinieerd door zijn karakteristiek $q$. Die karakteristiek is bij cryptografische toepassingen doorgaans een groot priemgetal $p$ of 2, hoewel tegenwoordig ook onderzoek gedaan wordt naar toepassingen met een karakteristiek 3. Een veld zonder zijn nul element wordt aangeduid als
\[\mathbb{F}_q^* = \mathbb{F}_q / \{ 0 \}.\]

Volgens de kleine stelling van Fermat geldt in elk eindig veld $a^q = a$. Van deze gelijkheid zal in het volgende hoofdstuk handig gebruikt gemaakt worden wanneer de inverse van een element moet berekend worden. Het voordeel van werken in een binair veld, m.a.w.\ $q = 2$, is dat optellingen en aftrekkingen equivalent zijn en zeer makkelijk uit te voeren zijn. Het is immers zo dat $1 + 1 = 2 \bmod 2 = 0$ en $0 - 1 = -1 \bmod 2 = 1$. Beiden kunnen dus berekend worden via een XOR operatie.

Verder kunnen extensies van graad $m$ van een veld gedefinieerd worden. In het geval $q = 2$ bekomt men dan een nieuw veld $\mathbb{F}_{2^m}$. Er dient dan ook een reductie veelterm $P$ opgegeven te worden. Een extensieveld wordt gedefinieerd als:
\[\mathbb{F}_{q^m} \cong \mathbb{F}_q [z] / (P). \]

\paragraph{Voorbeeld} Om de constructie van een extensieveld enigszins te verduidelijken wordt een klein voorbeeld gegeven. Er wordt gewerkt in karakteristiek $q = 2$. Stel $P = z^2 + z + 1$. Het extensieveld is dus gedefinieerd als:
\[\mathbb{F}_{2^2} \cong \mathbb{F}_2 [z] / (z^2 + z + 1). \]
Verder $A = z$ en $B = z + 1$. De resultaten van de optelling en vermenigvuldiging van $A$ en $B$ zijn dan respectievelijk:
\[\begin{aligned}
A + B &= 2z + 1 \qquad & A \cdot B &= z^2 + z\\
&= 1	& &= 2z + 1\\
\end{aligned}\]

\subsection{Elliptische krommen}

Een elliptische kromme $E$ wordt gevormd door de verzameling van punten die voldoen aan de vergelijking:
\[E: Y^2 Z + a_1 XYZ + a_2 Y Z^2 = X^3 + a_3 X^2 Z + a_4 X Z^2 + a_5 Z^3.\]
Het enige punt waarvoor $Z = 0$ en de vergelijking geldt ($X = 0,  Y = 1, Z = 0$), wordt het punt op oneindig $\mathcal{O}$ genoemd. Indien wordt gesteld dat $x = \frac{X}{Z}$ en $y = \frac{Y}{Z}$, bekomt men de affiene Weierstrass vergelijking:
\[E: y^2 + a_1 xy + a_2 y = x^3 + a_3 x^2 + a_4 x + a_5.\]
Merk op dat in deze vorm het punt $\mathcal{O}$ niet meer voldoet aan de vergelijking, ook al behoort het nog steeds tot de kromme. De kromme dient zo gedefinieerd te zijn dat $\forall A \in E$ de parti\"ele afgeleiden $\frac{\partial P}{\partial X}$, $\frac{\partial P}{\partial Y}$ en $\frac{\partial P}{\partial Z}$ nooit allen tegelijkertijd gelijk zijn aan nul.

De natuurlijke bewerking op een kromme is de ``tangent-and-chord'' methode, die wordt weergegeven in \reffig{figuur-pairings-tangent-and-chord}. De bewerking wordt additief geschreven en heeft als neutral element het punt op oneindig $\mathcal{O}$. Afhankelijk van het veld waarover de kromme gedefinieerd is, zullen de formules om de ``tangent-and-chord'' methode uit te voeren anders zijn. 

\begin{figure}[h]
	\centering
		\includegraphics[width=8cm]{tangent-and-add}
		\caption{``tangent-and-add'' methode op een elliptische kromme\label{figuur-pairings-tangent-and-chord}}
\end{figure}

Aan de hand van de vorige bewerking kan een scalaire vermenigvuldiging vastgelegd worden, met $a \in \mathbb{Z}$:
\[\begin{aligned}
a \cdot A	&= A + \ldots + A\\
0 \cdot A	&= \mathcal{O}\\
-a \cdot A	&= a \cdot -A
\end{aligned}\]

De orde $o$ van een punt $A$ op de kromme is gelijk aan de minimum waarde waarvoor $o \cdot A = \mathcal{O}$. Het is mogelijk dat $o = \infty$. Van alle punten waarvoor $o$ een deler is van $n$ wordt gezegd dat ze in de $n$-torsiesubgroep van $E$ zitten. Zo een subgroep wordt genoteerd als:
\[E[n] = \{ A \in E : n \cdot A = \mathcal{O} \}.\]

Het aantal punten $\#E$ op $E$ wordt de orde van de kromme genoemd. Voor een kromme over een veld $\mathbb{F}_q$ is $\#E = q + 1 - t$, met $t$ de ``trace'' van de kromme. Indien $q \mid t$ wordt de kromme supersingulier genoemd. Voor bepaalde types krommen bestaat er een gesloten formule voor $\#E$.

Ten slotte wordt nog de inbeddingsgraad $k$ gedefinieerd als het kleinste gehele getal waarvoor $n \mid q^k - 1$.

\section{Definitie van pairings}

Een pairing is in dit geval een functie $f(A, B)$ met als argumenten twee punten uit een additieve groep en als resultaat een punt in een multiplicatieve groep:
\[f(A, B): \mathbb{G}_1 \times \mathbb{G}_2 \rightarrow \mathbb{G}_T.\]
Een pairing moet tevens volgende eigenschappen bezitten: bilineariteit, non-degeneratie en moet goed gedefinieerd zijn. De betekenis van deze drie begrippen is:

\begin{enumerate}
	\item Bilineariteit: $\forall A_1, A_2, A \in \mathbb{G}_1$ en $\forall B_1, B_2, B \in \mathbb{G}_2$ geldt $f(A_1 + A_2, B) \equiv f(A_1, B) \cdot f(A_2, B)$ en $f(A, B_1 + B_2) \equiv f(A, B_1) \cdot f(A, B_2)$.

	\item Non-degeneratief: $\forall A \in \mathbb{G}_1, \: \exists B \in \mathbb{G}_2$ waarvoor $f(A, B) \neq 1$.

	\item Goed gedefinieerd: $\forall B \in \mathbb{G}_2$ is $f(A, B) = 1$ als en slechts als $A = \mathcal{O}$.
\end{enumerate}

Het zijn de drie eigenschappen waaraan pairings moeten voldoen die het zo moeilijk maken ze te genereren. Ten tijde van dit schrijven waren onder meer volgende pairings bekend: Weil, Tate, $\eta_T$ \cite{eta} en Ate \cite{ate}. In deze thesis zal gewerkt worden met de Tate pairing, waarvan bewezen is dat ze effici\"enter te berekenen is dan de Weil pairing.

De vermelde $\eta_T$ en Ate pairing zijn variaties op de Tate pairing die gebruikt kunnen worden indien voor specifieke elliptische krommen gekozen wordt. Indien aan de juiste voorwaarden voldaan wordt, zal de schakeling die in het volgende hoofdstuk wordt voorgesteld dus ook gebruikt kunnen worden om deze pairings te berekenen.

De definitie van de Tate pairing over elliptische krommen is als volgt:
\[e(A, B): E(\mathbb{F}_{q^k})[n] \times E(\mathbb{F}_{q^k})/n \cdot E(\mathbb{F}_{q^k}) \rightarrow \mathbb{F}_{q^k}^* / (\mathbb{F}_{q^k}^*)^n\]
Het resultaat van de Tate pairing is niet uniek, maar een element van een equivalentieklasse in $\mathbb{F}_{q^k}^* / (\mathbb{F}_{q^k}^*)^n$. Voor twee resultaten $a, b \in \mathbb{F}_{q^k}^*$ geldt $a \equiv b$ als en slechts als er een element $c \in \mathbb{F}_{q^k}^*$ waarvoor $a = bc^n$. Aangezien voor cryptografische toepassingen een uniek resultaat gewenst zal zijn, dient die $n$-de macht weggewerkt te worden. Dit kan door het resultaat te verheffen tot de macht $\frac{q^k - 1 }{n}$. Aangezien $a^{q^k - 1} = 1$ voor elke $a \in \mathbb{F}_{q^k}^*$ (kleine stelling van Fermat), verkrijgt men dan dus een $n$-eenheidswortel.

\section{Berekening van de Tate pairing}

Reeds in 1986 stelde Miller een methode voor om de Tate pairing te berekenen \cite{miller, barreto-efficient}. \refalg{algoritme-pairings-miller} geeft weer hoe dat algoritme  in z'n werk gaat.
        % Wiskundige informatie over pairings 

\Chapter{Implementatie}\label{hfdst-implementatie}

In dit hoofdstuk wordt de implementatie van een schakeling voor de berekening van de Tate pairing uit de doeken gedaan. Er zal onderzocht worden welke basisbewerkingen nodig zijn en hoe deze verwezenlijkt kunnen worden in hardware. Vervolgens wordt een schakeling ontworpen die aan de hand hiervan alle nodige berekeningen kan uitvoeren in het veld $\mathbb{F}_{2^m}$. Ten slotte is er dan nog de schakeling die alle berekeningen voor het Miller algoritme in goede banen leidt. Allereerst wordt echter gekeken welke beperkingen aan de implementatie opgelegd moeten worden.

\section{Beperkingen}\label{sectie-implementatie-beperkingen}

Het doel is de uiteindelijke schakeling zo klein mogelijk te maken, zodat ze gebruikt kan worden in bv.\ netwerken van sensoren of smartcards. Beperking van de oppervlakte is dus de belangrijkste factor. Een tweede belangrijke factor is stroomverbruik, maar dat is helaas zeer moeilijk te berekenen. Het verbruik hangt echter samen met de oppervlakte, dus het beperken daarvan zal ook het verbruik ten goede komen. Het verbruik kan ook verlaagd worden door een lagere kloksnelheid voor de schakeling te gebruiken, wat uiteraard de rekensnelheid niet bevordert. De rekensnelheid is echter geen prioriteit en dus kan dit aspect bij het ontwerp van de schakelingen genegeerd worden. Op dit alles zal dieper ingegaan worden in \refhfdst{hfdst-resultaten}

Algemeen kan gesteld worden dat hoe kleiner het uiteindelijke resultaat is, hoe beter. Het is dus cruciaal de elementen te identificeren die het meeste plaats innemen in een ASIC schakeling. In \reftbl{tabel-implementatie-beperkingen-elementen-gatecount} is de grootte van de belangrijkste elementen te vinden. Deze cijfers gelden enkel bij gebruik van $0.13 nm$ low leakage technologie. De ordening van de elementen blijft echter behouden voor andere technologi\"en. Uit de tabel blijkt dat het gebruik van flip-flops (registers), adders en multiplexers zoveel mogelijk beperkt moet worden.

\begin{table}[h]
	\caption{Grootte van elementen in een ASIC schakeling in gates$/$bit ($0.13 nm$ low leakage technologie)\cite{cell-databook}}
	\label{tabel-implementatie-beperkingen-elementen-gatecount}
	\begin{tabular}{|l|r|}
		\hline
		Element			& Gates$/$bit\\
		\hline
		D flip-flop met reset	& 6\\
		D flip-flop zonder reset	& 5.5\\
		D latch			& 4.25\\
		full adder		& 5.5\\
		3 ingang MUX	& 4\\
		2 ingang XNOR	& 3.75\\
		2 ingang XOR	& 3.75\\
		2 ingang MUX	& 2.25\\
		2 ingang OR		& 1.25\\
		2 ingang AND	& 1.25\\
		2 ingang NOR	& 1\\
		2 ingang NAND	& 1\\
		NOT				& 0.75\\
		\hline		
	\end{tabular}
\end{table}

\section{Modular Arithmetic Logical Unit}\label{sectie-implementatie-malu}

De kern van de hardware implementatie wordt gevormd door de Modular Arithmetic Logical Unit (MALU)\cite{sakiyama}. Dit circuit laat toe basis bewerkingen uit te voeren op getallen. Gezien de beperking die is opgelegd aan de oppervlakte van de schakeling, wordt enkel de optelling ge\"implementeerd. Later wordt met behulp daarvan elke andere nodige berekening verwezenlijkt.

Aangezien er in het veld $\mathbb{F}_{2^m}$ gewerkt wordt, is een optelling equivalent aan een XOR bewerking. De bewerking die moet uitgevoerd kunnen worden is:

\[\begin{aligned}
T + B	&= T \xor B\\
		&= R \mod P
\end{aligned}\]

Merk op dat bij een optelling de graad van $R$ enkel kleiner of gelijk kan zijn aan die van $T$ en $B$. Indien $B$ van graad $\leq m$ is en $T$ van graad $\leq m + 1$, is de modulo bewerking te implementeren als in \refalg{algoritme-implementatie-malu-modulo}.

\begin{algorithm}[h]
\caption{Modulo optelling in $\mathbb{F}_{2^m}$}
\label{algoritme-implementatie-malu-modulo}
	\KwIn{$B \in \mathbb{F}_{2^m}$, $T \in \mathbb{F}_{2^{m + 1}}$}
	\KwOut{$R \mod P \in \mathbb{F}_{2^m}$}
	\SetKwFunction{Degree}{Degree}

	$R \leftarrow T \xor B$\;

	\If{$\Degree{T} = m$}{
		$R \leftarrow R \xor P$\;
	}
\end{algorithm}

Een voor de hand liggende schakeling die dit alles implementeert, is te zien in \reffig{figuur-implementatie-malu-basic-noshift}. Ingang $P_{\text{in}}$ dient afhankelijk van $T_{m}$ ingesteld te worden op $0$ of $P$. $P_{m}$ kan genegeerd worden, aangezien in het resultaat de graad $< m$ is.

\begin{figure}[h]
	\begin{center}
		\includegraphics[width=12cm]{malu-basic-noshift}
		\figcaption{MALU - Basis ontwerp}\label{figuur-implementatie-malu-basic-noshift}
	\end{center}
\end{figure}

In \refsect{sectie-implementatie-gf2m} zal blijken dat het vaak nodig zal zijn om het resultaat $R$ te vermenigvuldigen met $z$, maw.\ alle bits 1 plaats naar links te verschuiven. Indien die bewerking wordt toegevoegd, bekomt men de schakeling uit \reffig{figuur-implementatie-malu-basic}. Net als de vorige implementatie bestaat deze uit $2m$ XOR poorten.

\begin{figure}[h]
	\begin{center}
		\includegraphics[width=12cm]{malu-basic}
		\figcaption{MALU - Basis ontwerp met shift}\label{figuur-implementatie-malu-basic}
	\end{center}
\end{figure}

Aangezien voor het ontwerp het veld en de modulo veelterm op voorhand bepaald zijn, is het mogelijk een zeer groot aantal XOR poorten uit het ontwerp te verwijderen. De ingang en de bijhorende $m$ XOR poorten kunnen vervangen worden door een 1 bit `modulo enable' ingang $mod_{\text{in}}$ en er worden enkel XOR poorten geplaatst voor de bits $i$ waarvoor $P_i = 1$. Hierdoor wordt het aantal ingangen drastisch verkleind en worden 
\[\Delta = m - (\text{hamm}(P) - 1)\]
XOR poorten uitgespaard, met hamm$(P)$ gelijk aan het Hamming gewicht van de binaire representatie van $P$.

In dit geval is $m = 163$ en $P = z^{163} + z^7 + z^6 + z^3 + 1$. Er zijn dus $\text{hamm}(P) - 1 = 4$ XOR poorten nodig, wat een besparing van $163 - 4 =  159$ XOR poorten oplevert ($51\%$ kleiner dan de oorsponkelijke grootte).

De resulterende schakeling is te zien in \reffig{figuur-implementatie-malu-optimized}.

\begin{figure}[h]
	\begin{center}
		\includegraphics[width=12cm]{malu-optimized}
		\figcaption{MALU - Geoptimaliseerd ontwerp met shift}\label{figuur-implementatie-malu-optimized}
	\end{center}
\end{figure}

\section{Berekeningen in $\mathbb{F}_{2^m}$}\label{sectie-implementatie-gf2m}

\subsection{Basisontwerp}\label{subsectie-implementatie-gf2m-basisontwerp}

De eerder ontworpen MALU schakeling laat toe optellingen te doen, maar het Miller algoritme vereist dat er ook vermenigvuldigingen worden uitgerekend. Delingen en machtsverheffingen kunnen met behulp van vermenigvuldiging berekend worden en dienen dus niet rechtstreeks ge\"implementeerd te worden. Indien dus zowel optellingen als vermenigvuldigingen berekend kunnen worden, is alles voorhanden om de Tate pairing te berekenen.

Door toepassing van een ``shift and add'' algoritme, kan de waarde van \mbox{$A \cdot B = R$} berekend worden met behulp van de MALU schakeling. In \refalg{algoritme-implementatie-gf2m-multiply} is te zien hoe dit juist in z'n werk gaat. Door de modulo operatie telkens op het tussenresultaat uit te voeren, is het steeds van graad $\leq m$ en kan het opgeslagen worden in $T$. Op het einde moet het resultaat door $z$ gedeeld worden, wat neerkomt op een verschuiving van alle bits met 1 plaats naar rechts.

\begin{algorithm}[h]
\caption{``Shift and add'' vermenigvuldiging in $\mathbb{F}_{2^m}$}
\label{algoritme-implementatie-gf2m-multiply}
	\KwIn{$A, B \in \mathbb{F}_{2^m}/[P]$}
	\KwOut{$R \in \mathbb{F}_{2^m}/[P]$}
	\KwData{$T \in \mathbb{F}_{2^{m + 1}}$}
	\SetKwFunction{Degree}{Degree}

	$T \leftarrow 0$\;
	\For{$i \leftarrow m - 1$ \KwTo $0$}{
		\eIf{$A_i = 1$}{
			$b \leftarrow B$\;
		}{
			$b \leftarrow 0$\;
		}
	
		$T \leftarrow T \xor b$\;
	
		\If{$\Degree{T} = m$}{
			$T \leftarrow T \xor P$\;
		}
		$T \leftarrow T \ll 1$\;
	}
	$R \leftarrow T \gg 1$\;
\end{algorithm}

Wanneer de optelling en vermenigvuldiging nu in een schakeling gegoten worden, is het noodzakelijk een onderscheid te kunnen maken tussen beide bewerkingen. Ook moet kunnen aangegeven worden wanneer de berekening klaar is. Ten slotte moet het resultaat opgeslagen kunnen worden, zodat de uitgang van de schakeling correct blijft. Omdat in het Miller algoritme verscheidene keren de som $R + 1$ moet berekend worden, wordt ook een ingang $plus\_one$ voorzien. De uiteindelijke schakeling is te zien in \reffig{figuur-implementatie-wrapper-gf2m}. Het register $cycle$ is $\log _2 (m)$ bits groot en $T$ is $m$ bits. De waarde van $F_m$ wordt opgeslagen in register $mod$. Alle overige registers zijn 1 bit groot.

\begin{figure}[h]
	\begin{center}
		\includegraphics[width=12cm]{wrapper-gf2m}
		\figcaption{Schakeling voor berekeningen in $\mathbb{F}_{2^m}$}\label{figuur-implementatie-wrapper-gf2m}
	\end{center}
\end{figure}

Gezien de eenvoud van de schakeling is het niet nodig een FSM te implementeren, de besturing kan volledig via logica gebeuren. Die logica wordt getoond in \reffig{figuur-implementatie-wrapper-gf2m-logica}.

\begin{figure}[h]
	\begin{center}
		\includegraphics[width=12cm]{wrapper-gf2m-logica}
		\figcaption{Logica voor besturing van de schakeling voor berekeningen in $\mathbb{F}_{2^m}$}\label{figuur-implementatie-wrapper-gf2m-logica}
	\end{center}
\end{figure}

%Indien dit algoritme in een schakeling gegoten wordt, dienen enkele toevoegingen te gebeuren. Omdat een vermenigvuldiging langer dan \'e\'en klokslag duurt, is het noodzakelijk een $start$ ingang en $ready$ uitgang te hebben. Ook moet worden bijgehouden of $A$ en $B$ opgeteld of vermenigvuldigd dienen te worden. Deze functie wordt vervuld door het register $mode$. Ten slotte moet het mogelijk zijn om $R$ in $T$ op te slaan, zodat het juiste resultaat niet verloren gaat.

%Bij het hoog gaan van $start$, wordt nagegaan wat de waarde van $mode$ is. Indien het om een optelling gaat, wordt $A$ opgeslagen in $T$, anders wordt $T \leftarrow 0$ zoals in \refalg{algoritme-implementatie-gf2m-multiply}. Bij een optelling is het resultaat na \'e\'en klokslag beschikbaar aan de uitgang van het MALU blok. Het dient dan enkel nog 1 bit naar rechts verschoven te worden, net zoals moet gebeuren op het einde van een vermenigvuldiging.

%Wanneer dit alles in rekening gebracht wordt, verkrijgt men uiteindelijk de schakeling in \reffig{figuur-implementatie-wrapper-gf2m}. Om het geheel overzichtelijker te houden, bevat register $T$ een getal in $\mathbb{F}_{2^m}$ en wordt de waarde van $T_m$ bijgehouden in het register $mod$.

\subsection{Versnelling van de vermenigvuldiging}\label{subsectie-implementatie-gf2m-versnelling}

Wanneer met behulp van de schakeling in \reffig{figuur-implementatie-wrapper-gf2m} een vermenigvuldiging wordt berekend, zal het $m$ klokcycles duren eer het resultaat beschikbaar is aan de uitgang. Het is echter mogelijk dat aantal drastisch naar beneden te halen door $d$ MALU's te gebruiken en dus $d$ optellingen per klokcycle uit te voeren. Het principe hiervan wordt ge\"illustreerd in \reffig{figuur-implementatie-wrapper-gf2m-d}.

\begin{figure}[h]
	\begin{center}
		\includegraphics[width=12cm]{wrapper-gf2m-d}
		\figcaption{Schakeling voor berekeningen in $\mathbb{F}_{2^m}$ met woordbreedte $d$}\label{figuur-implementatie-wrapper-gf2m-d}
	\end{center}
\end{figure}

De rekentijd van het Miller algoritme zal door toepassing van deze techniek gevoelig verkort kunnen worden. Hoe groter $m$ en hoe meer vermenigvuldigigen er uitgevoerd dienen te worden, des te significanter de tijdswinst die geboekt kan worden. Uiteraard gaat het gebruik van deze techniek wel in tegen de eerder opgelegde beperking aan de grootte van de uiteindelijke schakeling. Het is echter niet zo dat er enkel $d - 1$ extra MALU blokken dienen toegevoegd te worden, afhankelijk van $d$ en $m$ dient ook een extra multiplexer in de schakeling gestoken te worden. Dit is zoals opgemerkt in \refsect{sectie-implementatie-beperkingen} een zeer slechte zaak voor de  oppervlakte.

Stel bijvoorbeeld $d = 4$ (en $m = 	163$). Het resultaat van een optelling zal net zoals in het standaard ontwerp aanwezig aan de uitgang van MALU \nr 1. Het resultaat van een vermenigvuldiging zal echter aan de uitgang van MALU \nr 3 verschijnen, aangezien $163 \bmod 4 = 3$. Het eindresultaat dat in $T$ dient opgeslagen te worden, is voor een vermenigvuldiging dus
\[T = mod3 \text{ \# } T3_{162:1}\]
, terwijl dit voor een optelling
\[T = mod1 \text{ \# } T1_{162:1}\]
is. Met andere woorden, er dient nu niet enkel gekozen te kunnen worden voor de ingang $A$, $T_{\text{out}}$ of $T1_{\text{ready}}$, maar ook voor $T3_{\text{ready}}$.

Indien men toch wenst het vermenigvuldigen te versnellen, is het aangeraden een $d$ te kiezen waarvoor
\[m \bmod d = 1\]

Als voorbeeld worden enkele voor de hand liggende en optimale keuzes vergeleken voor $d$ indien $m = 163$ in \reftbl{tabel-implementatie-woordbreedte-d}.

\begin{table}[h]
	\caption{Voor de hand liggende versus optimale waarden voor woordbreedte $d$ indien \mbox{$m=163$}}
	\label{tabel-implementatie-woordbreedte-d}
	\begin{tabular}{|l|c|c|c|c|c|c|}
		\hline
		\multicolumn{7}{|c|}{Voor de hand liggende waarden voor $d$}\\
		\hline
		$d$			& 2	& 4	& 8	& 16	& 32	& 64\\
		$m \bmod d$	& 1	& 3	& 3	& 3	& 3	& 35\\
		\hline
		\multicolumn{7}{c}{}\\
		\hline
		\multicolumn{7}{|c|}{Ideale waarden voor $d$}\\
		\hline
		$d$			& 2	& 3	& 6	& 9	& 18	& 27\\
		$m \bmod d$	& 1	& 1	& 1	& 1	& 1	& 1\\
		\hline
	\end{tabular}
\end{table}

\section{Controller voor het Miller algoritme}\label{sectie-implementatie-miller}


\subsection{Inleiding}\label{subsectie-implementatie-miller-inleiding}

Nu een schakeling voorhanden is die toelaat alle benodigde berekeningen uit te voeren, rest nog een schakeling te ontwerpen die het Miller algoritme (\refalg{algoritme-cryptografie-pairings-miller}) uitvoert. Het algoritme met invulling van de gekende parameters, zonder uitwerking van de berekeningen, wordt gegeven in \refalg{algoritme-implementatie-miller-algemeen}.

\begin{algorithm}[h]
	\caption{Miller algoritme voor berekening van de Tate pairing - Algemene versie}
	\label{algoritme-implementatie-miller-algemeen}
	\KwIn{$P, Q \in E(\mathbb{F}_{2^{163}})[l]$}
	\KwOut{e($P, Q$)}
	\KwData{$V \in E(\mathbb{F}_{2^{163}})[l]$; $F, G \in \mathbb{F}_{2^{4 \cdot 163}}$}
	$F \leftarrow 1$\;
	$V \leftarrow P$\;
	\For{$i \leftarrow 162$ \KwTo $0$}{
		$F \leftarrow F^2 \cdot G_{V,V}(\phi(Q))$\;\nllabel{lijn-implementatie-miller-algemeen-double-1}
		$V \leftarrow 2V$\;\nllabel{lijn-implementatie-miller-algemeen-double-2}
		\If{$i = 82$}{\nllabel{lijn-implementatie-miller-algemeen-add-if}
			$F \leftarrow F \cdot G_{V,P}(\phi(Q))$\;\nllabel{lijn-implementatie-miller-algemeen-add-1}
			$V \leftarrow V + P$\;\nllabel{lijn-implementatie-miller-algemeen-add-2}
		}
	}
	$F \leftarrow F^{\frac{2^{4 \cdot 163} - 1}{l}}$\;
	\KwRet{$F$}
\end{algorithm}

Merk op dat op lijn~\ref{lijn-implementatie-miller-algemeen-add-if} slechts \'e\'en waarde moet nagekeken worden, aangezien $l = 2^{163} + 2^{82} + 1$.

Om \refalg{algoritme-implementatie-miller-algemeen} te kunnen implementeren, moeten eerst de verschillende berekeningen uitgewerkt worden. Vervolgens zal aan de hand van die uitwerkingen bepaald worden welke registers de uiteindelijke schakeling ten minste nodig heeft en ten slotte zal een FSM ontworpen worden.

\subsection{Uitwerking berekeningen}\label{subsectie-implementatie-miller-uitwerking}

Grofweg kan het algoritme opgedeeld worden in een verdubbelingsstap, een optellingstap, de vermenigvuldiging $F \cdot G$ en een finale exponentiatie. Elk van deze stappen zal verder uitgediept worden en er zal voor elke berekening bepaald worden hoeveel tussenresultaten minimum opgeslagen moeten worden. Het zal blijken dat voor de optellingstap een inversie in $\mathbb{F}_{2^m}$ uitgerekend moet worden. Uiteraard wordt ook dit verder onderzocht.

Bij elk van de volgende algoritmen zal aangegeven worden hoeveel en welke bewerkingen juist nodig zijn. Daarbij staat \textsf{A} voor een optelling, \textsf{M} voor een vermenigvuldiging, \textsf{S} voor een kwadratering en \textsf{I} voor een inversie. Aangezien er echter geen afzonderlijke schakeling voor kwadrateren ontworpen is, zijn \textsf{S} en \textsf{M} qua rekentijd in dit geval equivalent aan elkaar. De bewerking $a + 1$ neemt geen extra tijd in beslag, omdat die functie parallel met een optelling of vermenigvuldiging kan uitgevoerd worden door de $plus\_one$ van de vorige schakeling hoog te maken bij de start van een berekening.

\subsubsection{Verdubbelingsstap}

De verdubbelingstap wordt gevormd door lijnen \ref{lijn-implementatie-miller-algemeen-double-1} en \ref{lijn-implementatie-miller-algemeen-double-2} in \refalg{algoritme-implementatie-miller-algemeen}. Voor een hyperelliptische kromme worden de berekeningen gegeven in \refalg{algoritme-implementatie-miller-double}\cite{bertoni}.

\begin{algorithm}[h]
	\dontprintsemicolon
	\caption{Verdubbelingstap voor hyperelliptische krommen in het Miller algoritme}
	\label{algoritme-implementatie-miller-double}
	\KwIn{$x_V, y_V \in E(\mathbb{F}_{2^m})$}
	\KwOut{$x_{2V}, y_{2V} \in E(\mathbb{F}_{2^m})$; $G \in \mathbb{F}_{2^{4m}}$}
	\KwData{$\lambda \in \mathbb{F}_{2^m}$}
	$\lambda \leftarrow x_V^2 + 1$\comm{1 S}\;
	$x_{2V} \leftarrow \lambda ^2$\comm{1 S}\;
	$y_{2V} \leftarrow \lambda (x_{2V} + x_V) + y_V + 1$\comm{2 A, 1 M}\;\nllabel{lijn-implementatie-miller-double-yv}
	$G_{V,V}(\phi(Q)) \leftarrow \lambda (x_{\phi} + x_V) + (y_{\phi} + y_V)$\comm{4 A, 1 M}\;\nllabel{lijn-implementatie-miller-double-g}
\end{algorithm}

Lijn \ref{lijn-implementatie-miller-double-yv} in dit algoritme kan ook berekend worden als

\[\begin{aligned}
y_{2V}	&= y_V^4 + x_V^4\\
			&= (y_V + x_V)^4	
\end{aligned}\]

Aangezien dit echter 2 kwadrateringen en 1 optelling kost, wordt de voorkeur gegeven aan de eerste methode.

Door de specifieke vorm van $\phi(Q)$ kan lijn~\ref{lijn-implementatie-miller-double-g} uitgeschreven worden als

\[\begin{aligned}
	G_a	&=	\lambda (x_{\phi_a} + x_V) + (y_{\phi_a} + y_V)\qquad&	G_c	&= \lambda \cdot x_{\phi_c} + y_{\phi_c}\\
	G_b	&=	\lambda \cdot x_{\phi_b} + y_{\phi_b}&									&= 0\\
			&= \lambda + y_{\phi_b}&												G_d	&= \lambda \cdot x_{\phi_d} + y_{\phi_d}\\
			&=	\lambda + x_{\phi_a}&														&= 1\\
\end{aligned}\]

De variabele $G$ kan dus opgeslagen worden in twee registers van grootte $m$ in plaats van in vier.

Wanneer dit in rekening gebracht wordt en het algoritme op register niveau wordt uitgeschreven, bekomt men uiteindelijk \refalg{algoritme-implementatie-miller-double-detail}. Hierbij werd specifiek gelet op een minimum gebruik van tijdelijke registers.

%Om te achterhalen in welke volgorde de bewerkingen best worden uitgevoerd om zo weinig mogelijk tussenresultaten te moeten opslaan. Het resultaat hiervan staat in \reftbl{tabel-implementatie-miller-double}. Een gearceerde cel betekend dat de aangeduide variabele op dat moment ingelezen moet worden. Indien er voor een variabele onder een bepaald niveau geen gearceerde cellen meer staan, kan deze verwijderd worden.

%\begin{minipage}[t]{10cm}
\begin{algorithm}[h]
	\caption{Uitwerking van de verdubbelingstap voor hyperelliptische krommen in het Miller algoritme}
	\label{algoritme-implementatie-miller-double-detail}
	\KwIn{$x_V, y_V \in E(\mathbb{F}_{2^m})$}
	\KwOut{$x_{2V}, y_{2V} \in E(\mathbb{F}_{2^m})$; $G \in \mathbb{F}_{2^{4m}}$}
	\KwData{$\lambda \in \mathbb{F}_{2^m}$}
	$G_a \leftarrow x_V$; $G_b \leftarrow y_V$\;
	$\lambda \leftarrow G_a^2 + 1$; $x_{2V} \leftarrow \lambda ^2$\comm{2 S}\;
	$y_{2V} \leftarrow x_{2V} + G_a$; $y_{2V} \leftarrow y_{2V} \cdot \lambda$\comm{1 A, 1 M}\;
	$y_{2V} \leftarrow y_{2V} + G_b + 1$\comm{1 A}\;
	$G_a \leftarrow G_a + x_{\phi_a}$; $G_a \leftarrow G_a \cdot \lambda$\comm{1 A, 1 M}\;
	$G_a \leftarrow G_a + y_{\phi_a}$; $G_a \leftarrow G_a + G_b$\comm{2 A}\;
	$G_b \leftarrow \lambda + x_{\phi_a}$\comm{1 A}\;
\end{algorithm}
%\end{minipage}
%\begin{minipage}[b]{3cm}
%	{\renewcommand{\arraystretch}{0.9}
%	\begin{tabular}{|c|c|c|}
%		\hline
%		$x_V$	& $y_V$	& $\lambda$\\
%		\hline
%		\shadecell & \shadecell & \\
%		& & \shadecell \\
%		& & \\
%		& & \shadecell \\
%		& & \\
%		& & \\
%		& & \shadecell \\
%		& & \\
%		& & \\
%		& & \shadecell \\
%		\hline
%	\end{tabular}}
%\end{minipage}

Buiten registers voor $x_{2V}$, $y_{2V}$, $x_{\phi_a}$, $y_{\phi_a}$, $G_a$ en $G_b$ is er dus ook nog een register nodig om $\lambda$ in op te slaan.

\subsubsection{Optellingstap}

De optellingstap bestaat uit lijnen \ref{lijn-implementatie-miller-algemeen-add-1} en \ref{lijn-implementatie-miller-algemeen-add-2} van \refalg{algoritme-implementatie-miller-algemeen}. Voor een hyperelliptische kromme dienen de bewerkingen in \refalg{algoritme-implementatie-miller-add} uitgevoerd te worden\cite{bertoni}.

\begin{algorithm}[h]
	\caption{Optellingstap voor hyperelliptische krommen in het Miller algoritme}
	\label{algoritme-implementatie-miller-add}
	\KwIn{$x_V, y_V, x_P, y_P \in E(\mathbb{F}_{2^m})$}
	\KwOut{$x_{V + P}, y_{V + P} \in E(\mathbb{F}_{2^m})$; $G \in \mathbb{F}_{2^{4m}}$}
	\KwData{$\lambda \in \mathbb{F}_{2^m}$}
	$\lambda \leftarrow \frac{y_V + y_P}{x_V + x_P}$\comm{2 A, 1 I, 1 M}\;
	$x_{V + P} \leftarrow \lambda ^2 + x_V + x_P$\comm{2 A, 1 S}\;
	$y_{V + P} \leftarrow \lambda (x_{V+P} + x_P) + y_P + 1$\comm{2 A, 1 M}\;
	$G_{V,V}(\phi(Q)) \leftarrow \lambda (x_{\phi} + x_P) + (y_{\phi} + y_P)$\comm{4 A, 1 M}\;
\end{algorithm}

Net zoals bij de verdubbelingstap kan $G$ hier in 2 variabelen opgeslagen worden. Hoewel de optellingstap slechts \'e\'en maal moet worden uitgevoerd, is het uiteraard cruciaal dat ook hier zo weinig mogelijk tijdelijke variabelen gebruikt worden. Op die manier blijft de grootte van de uiteindelijke schakeling het kleinst. De uitgewerkte versie van het algoritme wordt gegeven in \refalg{algoritme-implementatie-miller-add-detail}. De meest tijdrovende stap hier is de inversie, waar in het volgende deel verder op ingegaan zal worden.

\begin{algorithm}[h]
	\caption{Uitwerking van de optellingstap voor hyperelliptische krommen in het Miller algoritme}
	\label{algoritme-implementatie-miller-add-detail}
	\KwIn{$x_V, y_V, x_P, y_P \in E(\mathbb{F}_{2^m})$}
	\KwOut{$x_{V + P}, y_{V + P} \in E(\mathbb{F}_{2^m})$; $G \in \mathbb{F}_{2^{4m}}$}
	\KwData{$\lambda, a \in \mathbb{F}_{2^m}$}
	$G_a \leftarrow x_V$; $G_b \leftarrow y_V$\;
	$\lambda \leftarrow G_a + x_P$; $\lambda \leftarrow \lambda^{-1}$\comm{1 A, 1 I}\;
	$a \leftarrow G_b + y_P$; $\lambda \leftarrow \lambda \cdot a$\comm{1 A, 1 M}\;
	$x_{V + P} \leftarrow \lambda ^2 + G_a$; $x_{V + P} \leftarrow x_{V + P} + x_P$\comm{2 A, 1 S}\;
	$y_{V + P} \leftarrow x_{V + P} + x_P$; $y_{V + P} \leftarrow y_{V + P} \cdot \lambda$\comm{1 A, 1 M}\;
	$y_{V + P} \leftarrow y_{V + P} + y_P + 1$\comm{1 A}\;
	$G_a \leftarrow x_{\phi_a} + x_P$; $G_a \leftarrow G_a \cdot \lambda$\comm{1 A, 1 M}\;
	$G_a \leftarrow G_a + y_{\phi_a}$; $G_a \leftarrow G_a + y_P$\comm{2 A}\;
	$G_b \leftarrow \lambda + x_{\phi_a}$\comm{1 A}\;
\end{algorithm}

In tegenstelling tot de verdubbelingstap zijn hier twee tijdelijke registers nodig, een voor $\lambda$ en een voor $a$. Verder zijn er twee registers nodig voor $x_P$ en $y_P$.

\subsubsection{Inversie}

Zoals reeds vermeld in \refsect{sectie-cryptografie-gf}, kan een inversie in een Galois veld berekend worden door toepassing van Fermats kleine theorema:

\[\begin{aligned}
a^{2^m}		&= a\\
a^{2^m - 1}	&= 1\\
a^{2^m - 2}	&= a^{-1}\\
\end{aligned}\]

De naieve manier om dit te berekenen zou zijn om $a$ $2^m - 2$ keer met zichzelf te vermenigvuldigingen. In dit geval zou dat betekenen dat er $2^{163} -2 = 11 692 013 098 647 223 345 629 478 661 730 264 157 247 460 343 806$ vermenigvuldigingen zouden moeten uitgevoerd worden. Zoiets is uiteraard onhaalbaar.

Een tweede manier bestaat er in de exponent te ontbinden in machten van 2 en 3. In dat geval zouden er nog 237 vermenigvuldigen nodig zijn.

Er is echter een derde, optimale manier die toegepast kan worden indien de exponent van de vorm $2^m - 2$ is. Dit gaat als volgt in zijn werk:

\[a^{2^m - 2} = (a^{2^{m - 1} - 1})^2\]

Als wordt aangenomen dat $m$ oneven is, is de macht van twee na het gelijkheidsteken dus even. Zolang de macht van twee even is, kan recursief volgende formule toegepast worden:

\[a^{2^i - 1} = (a^{2^{\frac{i}{2}} - 1})^{2^{\frac{i}{2}}} \cdot a^{2^{\frac{i}{2}} - 1}\]

Indien $a$ oneven is, dient volgende formule toegepast te worden:

\[a^{2^i - 1} = (a^{2^{i - 1} - 1})^2 \cdot a\]

Uiteindelijk eindigd men dan bij $a^2$ of $a^3$. Het totaal aantal bewerkingen is $m + 1$ kwadrateringen en $\lceil\log_2(m)\rceil + 1$ vermenigvuldigingen.

In het geval van $m = 163$ is de uiteindelijke keten zoals gegeven in \refalg{algoritme-implementatie-miller-inversie}.

\begin{algorithm}[h]
	\caption{Inversie in $\mathbb{F}_{2^{163}}$}
	\label{algoritme-implementatie-miller-inversie}
	\KwIn{$a \in \mathbb{F}_{2^{163}}$}
	\KwOut{$a^{-1} \in \mathbb{F}_{2^{163}}$}
	$a^3 \leftarrow a^2 \cdot a$\comm{1 S, 1 M}\;
	$a^{2^4 - 1} \leftarrow (a^3)^{2^2} \cdot a^3$\comm{2 S, 1 M}\;
	$a^{2^5 - 1} \leftarrow (a^{2^4 - 1})^2 \cdot a$\comm{1 S, 1 M}\;
	$a^{2^{10} - 1} \leftarrow (a^{2^5 - 1})^{2^5} \cdot a^{2^5 - 1}$\comm{5 S, 1 M}\;
	$a^{2^{20} - 1} \leftarrow (a^{2^{10} - 1})^{2^{10}} \cdot a^{2^{10} - 1}$\comm{10 S, 1 M}\;
	$a^{2^{40} - 1} \leftarrow (a^{2^{20} - 1})^{2^{20}} \cdot a^{2^{20} - 1}$\comm{20 S, 1 M}\;
	$a^{2^{80} - 1} \leftarrow (a^{2^{40} - 1})^{2^{40}} \cdot a^{2^{40} - 1}$\comm{40 S, 1 M}\;
	$a^{2^{81} - 1} \leftarrow (a^{2^{80} - 1})^2 \cdot a$\comm{1 S, 1 M}\;
	$a^{2^{162} - 1} \leftarrow (a^{2^{81} - 1})^{2^{81}} \cdot a^{2^{81} - 1}$\comm{81 S, 1 M}\;
	$a^{-1} \leftarrow (a^{2^{162} - 1})^2$\comm{1 S}\;
\end{algorithm}

Het aantal berekeningen in dit geval is 162 kwadrateringen en 9 vermenigvuldigingen. Er is een register nodig om $a$ bij de houden en twee voor de tussenresultaten $a^{2^i - 1}$ en $(a^{2^i - 1})^{2^i}$.
		% Hardware implementatie
\Chapter{Resultaten}

Gelukkig is het verbruik evenredig met de oppervlakte en zal een kleinere schakeling dus automatisch ook een lager verbruik hebben. Het verbruik hangt echter ook af van de kloksnelheid en wel aan de hand van volgende formule:

\[P = V^2 \cdot C \cdot f\].

Een lage kloksnelheid zal dus bevorderend zijn voor een laag verbruik. Uiteraard brengt het verlagen van de kloksnelheid 

De snelheid is niet zozeer belangrijk, hoewel er uiteraard een bovenlimiet is aan de maximale rekentijd. Indien de schakeling bijvoorbeeld in een smartcard ingebouwd zou worden, is het noodzakelijk

      % Resultaten vd implementatie
\Chapter{Algemeen besluit}

\section{Besluit}

Deze thesis handelde over pairings, een recente ontwikkeling op gebied van cryptografie. Pairings laten identiteits-gebaseerde cryptografie toe, wat zeer interessant is voor bijvoorbeeld netwerken van sensoren.

Er werd onderzocht hoe de Tate pairing ge\"implementeerd kan worden in hardware. Meer specifiek werd de nadruk gelegd op een compacte implementatie die daarenboven nog eens zo weinig mogelijk vermogen verbruikte. Een implementatie van dat type zou toegepast kunnen worden in de eerder vermelde netwerken van sensoren of toestellen met een beperkt vermogen.

Verscheidene bestaande algoritmen werden aangepast en geoptimaliseerd zodat ze met een minimum gebruik aan geheugen uitgevoerd konden worden. Er werd een geheugenontwerp voorgesteld dat een goed compromis gaf tussen grootte en verbruik. Tevens werden verschillende oppervlakte- en vermogensbesparende technieken toegepast om het uiteindelijke circuit zo goed mogelijk aan de doelstellingen te laten voldoen. Door aanpassing van de kloksnelheid en het aantal gebruikte rekenschakelingen (MALUs) kunnen de parameters van het ontwerp aangepast worden aan de individuele vereisten van een toepassing.

Het resulterende ontwerp is uniek in zijn soort. Ten eerste is de ASIC implementatie met zijn <30k gates zeer klein. Verder is het mogelijk een verbruik te behalen van enkele tientallen nanowatt, indien rekensnelheid geen punt is. Ook aan hogere snelheiden blijft het verbruik zeer laag vergeleken met het enige andere gedocumenteerde ontwerp uit de literatuur. Ten tijde van dit schrijven waren nog geen andere ontwerpen gepubliceerd waarbij de nadruk op compactheid en laag vermogenverbruik lag.

\section{Toekomstig onderzoek}

Als toekomstig onderzoek kan het interessant zijn na te gaan of er nog significante optimalisaties aan het ontwerp mogelijk zijn. Daarbij is zou onderzocht moeten worden of de FSM verder te vereenvoudingen valt, aangezien die zowat de helft van de oppervlakte van de totale implementatie inneemt. Het zou ook nuttig zijn te onderzoeken wat de ideale plaatsing van variabelen in het geheugen is, om zo het verbruik en de rekentijd nog verder te verlagen. Ook dient nagegaan te worden waarom bij de synthese de oppervlakte van de schakeling niet kleiner wordt na het toepassen van clock gating technieken en hoe dit opgelost kan worden.

Verder zou ook onderzocht moeten worden in welke mate werken over een groter Galois veld de oppervlakte en het verbruik wijzigt. Ook de implementatie van andere types pairings kan interessant zijn. Zo bestaan er bijvoorbeeld de $\eta_T$ en de modified Tate pairing die beiden berekend kunnen worden in ongeveer de helft van de tijd nodig voor de berekening van de Tate pairing.

% Ten slotte is onderzoek naar het beveiligen van de pairing berekeningen tegen side-channel aanvallen een gebied waarin nog veel nuttige ontwikkelingen kunnen gebeuren.
       % Conclusie

\appendix                 % Start appendices

%\include{code}            % Implementatie code
%\include{debugging}       % Manier van debugging

% BibTex referenties
%\cleardoublepage
\addcontentsline{toc}{chapter}{Bibliografie}
\bibliography{references}


% Lege achterpagina
%\clearpage
%\mbox{~}
%\thispagestyle{empty}

\end{document}
