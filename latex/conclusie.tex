\Chapter{Conclusie \& toekomstig onderzoek}

\section{Conclusie}

Deze thesis handelde over pairings, een recente ontwikkeling op gebied van cryptografie die identiteits-gebaseerde cryptografie toelaat. Er werd onderzocht hoe de Tate pairing ge\"implementeerd kan worden in hardware. Meer specifiek werd de nadruk gelegd op een compacte implementatie die daarbovenop nog eens zo weinig mogelijk vermogen verbruikte. Een implementatie van dat type zou toegepast kunnen worden in netwerken van sensoren of toestellen met een beperkt vermogen.

Verscheidene bestaande algoritmen werden aangepast en geoptimaliseerd zodat ze met een minimum gebruik aan geheugen uitgevoerd konden worden. Er werd een geheugenontwerp voorgesteld dat een goed compromis gaf tussen grootte en verbruik. Tevens werden verschillende oppervlakte- en vermogensbesparende technieken toegepast om het uiteindelijke circuit zo goed mogelijk aan de doelstellingen te laten voldoen.

Het resulterende ontwerp is uniek in zijn soort; de enige tot nu toe gepubliceerde ontwerpen leggen allen de nadruk op snelheid. Het hier gepresenteerde ontwerp voldoet aan de verwachtingen.
