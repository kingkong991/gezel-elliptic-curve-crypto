\Chapter{Algemeen besluit}

\section{Besluit}

Deze thesis handelde over pairings, een recente ontwikkeling op gebied van cryptografie. Pairings laten identiteits-gebaseerde cryptografie toe, wat zeer interessant is voor bijvoorbeeld netwerken van sensoren.

Er werd onderzocht hoe de Tate pairing ge\"implementeerd kan worden in hardware. Meer specifiek werd de nadruk gelegd op een compacte implementatie die daarenboven nog eens zo weinig mogelijk vermogen verbruikte. Een implementatie van dat type zou toegepast kunnen worden in de eerder vermelde netwerken van sensoren of toestellen met een beperkt vermogen.

Verscheidene bestaande algoritmen werden aangepast en geoptimaliseerd zodat ze met een minimum gebruik aan geheugen uitgevoerd konden worden. Er werd een geheugenontwerp voorgesteld dat een goed compromis gaf tussen grootte en verbruik. Tevens werden verschillende oppervlakte- en vermogensbesparende technieken toegepast om het uiteindelijke circuit zo goed mogelijk aan de doelstellingen te laten voldoen. Door aanpassing van de kloksnelheid en het aantal gebruikte rekenschakelingen (MALUs) kunnen de parameters van het ontwerp aangepast worden aan de individuele vereisten van een toepassing.

Het resulterende ontwerp is uniek in zijn soort. Ten eerste is de ASIC implementatie met zijn $<$30k gates zeer klein. Verder is het mogelijk een verbruik te behalen van enkele tientallen nanowatt, indien rekensnelheid geen punt is. Ook aan hogere snelheiden blijft het verbruik zeer laag vergeleken met het enige andere gedocumenteerde ontwerp uit de literatuur. Ten tijde van dit schrijven waren nog geen andere ontwerpen gepubliceerd waarbij de nadruk op compactheid en laag vermogenverbruik lag.

\section{Toekomstig onderzoek}

Als toekomstig onderzoek kan het interessant zijn na te gaan of er nog significante optimalisaties aan het ontwerp mogelijk zijn. Daarbij zou onderzocht moeten worden of de FSM verder te vereenvoudingen valt, aangezien die zowat de helft van de oppervlakte van de totale implementatie in beslag neemt.

Het zou ook nuttig zijn te onderzoeken of het geheugenontwerp nog verder geoptimaliseerd kan worden. Een optimale plaatsing van variabelen in het geheugen zou helpen om het verbruik en de rekentijd nog verder te verlagen. Er dient ook nagegaan te worden waarom bij de synthese de oppervlakte van de schakeling niet kleiner wordt na het toepassen van clock gating technieken en hoe dit opgelost kan worden.

Verder zou ook onderzocht moeten worden in welke mate werken over een groter Galois veld de oppervlakte en het verbruik wijzigt. Mogelijk kunnen ook implementatie van andere types pairings, zoals de  de $\eta_T$ en de modified Tate pairing, nader bestudeerd worden.

% Ten slotte is onderzoek naar het beveiligen van de pairing berekeningen tegen side-channel aanvallen een gebied waarin nog veel nuttige ontwikkelingen kunnen gebeuren.
