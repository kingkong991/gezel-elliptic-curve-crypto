\chapter*{Samenvatting}

In deze thesis wordt de berekening van de Tate pairing onder de loep genomen. Een pairing is een wiskundige constructie waarvan in 2001 ontdekt werd dat ze gebruikt kan worden voor het implementeren van identiteits-gebaseerde cryptografie.

Er wordt een compact hardware ontwerp voorgesteld dat de Tate pairing over een supersinguliere curve in $\mathbb{F}_{2^{163}}$ op een zeer zuinige manier kan berekenen. Het geheugenverbruik van gebruikte algoritmes wordt geminimaliseerd en een ontwerp voor een effici\"ent geheugenblok passeert de revue. Ook wordt het effect van verscheidene vermogensoptimalisaties onderzocht.

De uiteindelijke schakeling neemt minimum ongeveer 30k gates in beslag, meer dan drie keer kleiner dan het kleinste ontwerp uit de literatuur. Een verbruik zo laag als 206 nanowatt kan bereikt worden, hoewel in dat geval de rekentijd 51.5 seconden bedraagt. Indien gewenst kan de berekening versneld worden door uitbreiding van de rekenkern. De energie-effici\"entie is, afhankelijk van gekozen parameters, tussen twee en twintig maal beter dan die van de enige andere implementatie uit de literatuur waarvoor dit berekend kon worden.

Ten tijde van dit schrijven zijn nog geen andere compacte, energie-effici\"ente ontwerpen gepubliceerd. Het voorgestelde ontwerp is in dat opzicht dus uniek.
